\documentclass{article}
\usepackage{tpack}

\title{ES704 - Instrumentação Básica}
\author{Guilherme Nunes Trofino}
\authorRA{217276}
\project{Resumo Teórico}

\begin{document}
    \maketitle
\newpage

    \tableofcontents
\newpage

\section{Introdução}
    \paragraph{Apresentação}Neste documento será descrito as informações necessárias para compreensão e solução de exercícios relacionados a disciplina \thetitle. Note que este documento são notas realizadas por \theauthor, em \today.

    \subsection{Medição}
        \paragraph{Definição}Atribuição do valor, ou tendência de valores, à variável de interesse, normalmente proviniente a um sistema que se deja analisar.

    \subsection{Calibração}
        \paragraph{Definição}Determinar matematicamente a relação entre a entrada e sistema de medição, possuindo \textbf{dimensões} que devem estar de acordo com as \textbf{Normas Técnicas}. Classificadas de acordo com mostrado a seguir:

        \subsubsection{Calibração Estática}
            \paragraph{Definição}Aplicar uma entrada conhecida para medir a resposta do sistema, representada em um gráfico como mostrado a seguir:

            \paragraph{Curva de Calibração}Relação entre a entrada e saída de dados definida por $y = f(x)$.

            \paragraph{Sensibilidade Estática}Representa a proporção entre a entrada e a saída nas regiões de baixa e alta sensibilidade como expressado pela seguinte equação:
                \begin{equation}
                    \boxed{
                        K = \diff{f(x)}{x}\Bigr|_{\substack{x = x_{0}}}
                    }
                \end{equation}

            \paragraph{Faixa Dinâmica}Intervalo no qual a curva de calibração é válida, pois há dados para suportar as hipóteses. Fora deste intervalo, a resposta será uma \textbf{Extrapolação}.
                \begin{equation}
                    \boxed{r_{i} = x_{\text{max}} - x_{\text{min}}}
                    \qquad
                    \boxed{r_{o} = y_{\text{max}} - y_{\text{min}}}
                \end{equation}

            \paragraph{Resolução}Menor incremento que pode ser detectado pelo sistema de medição. Geralmente determinado pela escala indicada pelos equipamentos de medição ou pelas restrições do sistema.

            \paragraph{Coeficiente de Determinação}Grau de compatibilidade entre os valores amostrados, $y$, e os valores do \textbf{Curve Fitting}, $y_{C}$, relacionados pela seguinte equação:
                \begin{equation}
                    \boxed{
                        R^{2} = 1 - \frac{\sum(y - y_{C})^{2}}{\sum(y - \bar{y})^{2}}
                    }
                \end{equation}
            Quanto mais próximo de 1 melhor será o ajuste da curva.

        \subsubsection{Calibração Dinâmica}
            \paragraph{Definição}Aplicar uma entrada conhecida para medir a resposta do sistema com a saída sempre seguindo a forma de onda da entrada.

            \paragraph{Constante de Tempo}Variável denotada pela constante $\tau$ para representar o tempo necessário para que o sistema atinga $63,2\%$ de seu valor final. Comumente, define-se que $5\tau$ representa o tempo necessário para que sistema se estabilize após um estímulo.

    \subsection{Erro}
        \paragraph{Definição}Diferença entre os valores medidos $y$ e seus correspondentes valores reais $y'$ de uma variável como expresso pela seguinte equação:
            \begin{equation}
                \boxed{
                    e = |y - y'|
                }
            \end{equation}

        \subsubsection{Erro Aleatório}
            \paragraph{Definição}Problemas que afetam a \textbf{Precisão} do sistema, pois causam o espalhamento dos dados em relação ao valor real.

        \subsubsection{Erro Sistemático}
            \paragraph{Definição}Problemas que afetam a \textbf{Exatidão} do sistema, pois causam a deflexão do valor medido em relação ao valor real.

        \subsubsection{Erro Experimental}
            \paragraph{Definição}Diferentes problemas visíveis durante a análise dos dados causados por métodos inadequados de experimentação, entre os principais estão:
                \begin{enumerate}[rightmargin = \leftmargin]
                    \item \textbf{Histerese:} Diferença na resposta do sistema exitado por entradas crescentes, \texttt{upscale} e decrescentes, \texttt{downscale};

                    \item \textbf{Linearidade:} Diferença na resposta em relação a curva de calibração linear;

                    \item \textbf{Sensibilidade:} Diferença no valor do ganho do sistema ;

                    \item \textbf{Retorno ao Zero:} Diferença na resposta do sistema para uma entrada nula;
                \end{enumerate}
\end{document}