\documentclass{article}
\usepackage{tpack}

\title{ES704 - Instrumentação Básica}
\author{Guilherme Nunes Trofino}
\authorRA{217276}
\project{Resumo Teórico}

\begin{document}
    \maketitle
\newpage

    \tableofcontents
\newpage

    \section{Introdução}
        \paragraph{Apresentação}Neste documento será descrito as informações necessárias para compreensão e solução de exercícios relacionados a disciplina \thetitle. Note que este documento são notas realizadas por \theauthor, em \today.

    \subsection{Medição}
        \paragraph{Definição}Atribuição do valor, ou tendência de valores, à variável de interesse, normalmente proviniente a um sistema que se deja analisar.

    \subsection{Calibração}
        \paragraph{Definição}Determinar matematicamente a relação entre a entrada e sistema de medição, possuindo \textbf{dimensões} que devem estar de acordo com as \textbf{Normas Técnicas}. Classificadas de acordo com mostrado a seguir:
            \begin{enumerate}[rightmargin = \leftmargin]
                \item \textbf{Calibração Estática:} Aplicar uma entrada conhecida para medir a resposta do sistema, representada em um gráfico;

                \item \textbf{Calibração Dinâmica:} Aplicar uma entrada continuamente, medindo a saída do sistema;
                    \begin{enumerate}[noitemsep, rightmargin = \leftmargin]
                        \item \texttt{Constante de Tempo:} Representa o tempo necessário para que a sistema perco

                        \item \texttt{Resposta na Frequência:} 
                    \end{enumerate}

               \item \textbf{Erros:}
                    
            \end{enumerate}


\end{document}