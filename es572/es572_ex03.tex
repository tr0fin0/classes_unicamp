\documentclass{article}
\usepackage{tpack}



\titleformat
    {\subsection}           % Part
    [block]                 % Part Shape
    {\normalfont\Large}     % Font Size
    {}                      % Label Numbering
    {0mm}                   % Part Separation
    {}                      % Code Before
    []                      % Code After

    \titlespacing*{\subsection}{0mm}{5mm}{2.5mm}


\title{ES572 - Circuitos Lógicos}
\author{Guilherme Nunes Trofino}
\authorRA{217276}
\project{Atividade Teórica}


\begin{document}
    \maketitle
\newpage

    \section{Atividade Teórica}
        \paragraph{Apresentação}Resolução das questões de Circuitos Lógicos por Guilherme Nunes Trofino, 217276, sobre \textbf{Sistemas de Numeração} e \textbf{Portas Lógicas}.

        \subsection{Questão 1}
            \begin{exercise}
                Babilônios desenvolveram os números sexagesimais, base 60, há mais de 4000 mil anos. Quantos bis de informação são descritos por um dígito sexagesimal? Converta $2021_{(10)}$.
            \end{exercise}
            \begin{resolution}
                Nota-se que a quantidade de bits de informação na base sexagesimal será obtido pela seguinte equação:
                    \begin{equation*}
                        \boxed{\log_{2}(60) \approx 5.9069}
                    \end{equation*}
                Na sequência nota-se que $2021_{(10)}$ poderá ser convertido através da \textbf{Divisão Sucessiva} como demonstrado abaixo:
                    \begin{figure}[H]
                        \centering
                        \basetenconversiontable{2021}{60}
                    \end{figure}
                Note que será necessário converter os valores numéricos para seus equivalentes alfabéticos, obtendo assim:
                    \begin{equation*}
                        \boxed{2021_{(10)} = Xf_{(60)}}
                    \end{equation*}
            \end{resolution}
\newpage

        \subsection{Questão 2}
            \begin{exercise}
                Realize as seguintes conversões:
                    \begin{table}[H]
                        \centering  
                        \begin{tabular}[]{crl}\hline
                            (1) & $0110111_{(2)}$   & $?_{(10)}$\\
                            (2) & $101101010_{(2)}$ & $?_{(16)}$\\
                            (3) & $C9_{(16)}$       & $?_{(8)}$\\
                            (4) & $A7_{(16)}$       & $?_{(10)}$\\
                            (5) & $743_{(10)}$      & $?_{(16)}$\\
                            (6) & $221_{(10)}$      & $?_{(2)}$\\\hline
                        \end{tabular}
                    \end{table}
            \end{exercise}
            \begin{resolution}
                As seguintes conversões devem ser realizadas:
                \begin{enumerate}[rightmargin = \leftmargin]
                    \item Nota-se que $0110111_{(2)}$ poderá ser convertido através de \textbf{Complemento}:
                        \begin{figure}[H]
                            \centering
                            \todecimal[2]{0110111}
                        \end{figure}

                    \item Nota-se que $101101010_{(2)}$ poderá ser convertido através de \textbf{Agrupamento}:
                        \begin{equation*}
                            \underbrace{\text{\textcolor{red}{000}1}}_{
                                \text{1}
                            }
                            \quad
                            \underbrace{\text{0110}}_{
                                \text{6}
                            }
                            \quad
                            \underbrace{\text{1010}}_{
                                \text{A}
                            } = 
                            \boxed{16A_{(16)}}
                        \end{equation*}

                    \item Nota-se que $C9_{(16)}$ poderá ser convertido através de \textbf{Agrupamento}:
                        \begin{equation*}
                            \underbrace{\text{C9}}_{
                                \underbrace{\text{\textcolor{red}{0}11}}_{
                                    \text{3}
                                }
                                \quad
                                \underbrace{\text{001}}_{
                                    \text{1}
                                }
                                \quad
                                \underbrace{\text{001}}_{
                                    \text{1}
                                }
                            } = 
                            \boxed{311_{(8)}}
                        \end{equation*}

                    \item Nota-se que $A7_{(16)}$ poderá ser convertido através de \textbf{Complemento}:
                        \begin{figure}[H]
                            \centering
                            \todecimal[16]{A7}
                        \end{figure}

                    \item Nota-se que $743_{(10)}$ poderá ser convertido através da \textbf{Divisão Sucessiva}:
                        \begin{figure}[H]
                            \centering
                            \basetenconversiontable{743}{16}
                        \end{figure}
                    Note que será necessário converter os valores numéricos para seus equivalentes alfabéticos, obtendo assim:
                        \begin{equation*}
                            \boxed{743_{(10)} = 2E7_{(16)}}
                        \end{equation*}

                    \item Nota-se que $221_{(10)}$ poderá ser convertido através da \textbf{Divisão Sucessiva}:
                        \begin{figure}[H]
                            \centering
                            \basetenconversiontable{221}{2}
                        \end{figure}
                \end{enumerate}
            \end{resolution}
\newpage

        \subsection{Questão 3}
            \begin{exercise}
                Determine os valores de um número de 12 bits nas seguintes configurações:
                    \begin{enumerate}[noitemsep]
                        \item Número sem sinal;
                        \item \href{https://en.wikipedia.org/wiki/Ones%27_complement}{Número em complemento de 1};
                        \item \href{https://en.wikipedia.org/wiki/Two%27s_complement}{Número em complemento de 2};
                        \item \href{https://en.wikipedia.org/wiki/Signed_number_representations}{Número em sinal-magnitude};
                    \end{enumerate}
            \end{exercise}
            \begin{resolution}
                Nota-se que os seguintes intervalos seriam possíveis:
                    \begin{table}[H]
                        \centering
                        \begin{tabular}[]{clcr}\hline
                            1. & ($0$          & , & $2^{N}-1$)\\
                            2. & ($-2^{N-1}+1$ & , & $2^{N-1}-1$)\\
                            3. & ($-2^{N-1}$   & , & $2^{N-1}-1$)\\
                            4. & ($-2^{N-1}+1$ & , & $2^{N-1}-1$)\\\hline
                        \end{tabular}
                    \end{table}
                Obtendo:
                    \begin{table}[H]
                        \centering
                        \begin{tabular}[]{clcr}\hline
                            1. & (    $0$ & , & $4095$)\\
                            2. & ($-2047$ & , & $2047$)\\
                            3. & ($-2048$ & , & $2047$)\\
                            4. & ($-2047$ & , & $2047$)\\\hline
                        \end{tabular}
                    \end{table}
            \end{resolution}
\newpage

        \subsection{Questão 4}
            \begin{exercise}
                Considere o número de 8 bits $11010011_{(2)}$ e represente-o através das seguintes codificações:
                    \begin{enumerate}[noitemsep]
                        \item Número sem sinal;
                        \item Número em sinal-magnitude;
                        \item Número em complemento de 1;
                        \item Número em complemento de 2;
                    \end{enumerate}
                Repita considerando o número de 9 bits $011010011_{(2)}$.
            \end{exercise}
            \begin{resolution}
                As seguintes conversões devem ser realizadas:
                \begin{enumerate}
                    \item Número sem sinal:
                        \begin{figure}[H]
                            \centering
                            \todecimal[2]{11010011}
                        \end{figure}

                    \item Número em sinal-magnitude:
                        \begin{figure}[H]
                            \centering
                            \todecimal[2]{1010011}
                        \end{figure}
                    Nota-se que o primeiro \texttt{bit} original era um 1, logo o número será $\boxed{-83_{(10)}}$.

                    \item Número em complemento de 1: Como o \texttt{LSB} é 1, inverte-se o número como mostrado:
                        \begin{table}[H]
                            \centering  
                            \begin{tabular}[]{cc cc cc cc}
                                $2^7$ & $2^6$ & $2^5$ & $2^4$ & $2^3$ & $2^2$ & $2^1$ & $2^0$\\\hline
                                128   & 64    & 32    & 16    & 8     & 4     & 2     & 1\\
                                  0   &  0    &  1    &  0    & 1     & 1     & 0     & 0\\
                            \end{tabular}
                        \end{table}
                    Nota-se que será necessário realizar a soma, logo o número será $\boxed{-44_{(10)}}$.

                    \item Número em complemento de 2;
                        \begin{table}[H]
                            \centering  
                            \begin{tabular}[]{cc cc cc cc}
                                $-2^7$ & $2^6$ & $2^5$ & $2^4$ & $2^3$ & $2^2$ & $2^1$ & $2^0$\\\hline
                                -128   & 64    & 32    & 16    & 8     & 4     & 2     & 1\\
                                   1   &  1    &  0    &  1    & 0     & 0     & 1     & 1\\
                            \end{tabular}
                        \end{table}
                    Nota-se que será necessário realizar a soma, logo o número será $\boxed{-45_{(10)}}$.
                \end{enumerate}
            \end{resolution}
\newpage
            \begin{resolution}
                As seguintes conversões devem ser realizadas:
                \begin{enumerate}
                    \item Número sem sinal:
                        \begin{figure}[H]
                            \centering
                            \todecimal[2]{011010011}
                        \end{figure}

                    \item Número em sinal-magnitude:
                        \begin{figure}[H]
                            \centering
                            \todecimal[2]{11010011}
                        \end{figure}
                    Nota-se que o primeiro \texttt{bit} original era um 0, logo o número será $\boxed{211_{(10)}}$.

                    \item Número em complemento de 1: Como o \texttt{LSB} é 0, não inverte-se o número como mostrado:
                        \begin{table}[H]
                            \centering  
                            \begin{tabular}[]{c cc cc cc cc}
                                $2^8$ & $2^7$ & $2^6$ & $2^5$ & $2^4$ & $2^3$ & $2^2$ & $2^1$ & $2^0$\\\hline
                                  256 & 128   & 64    & 32    & 16    & 8     & 4     & 2     & 1\\
                                    0 &   1   &  1    &  0    &  1    & 0     & 0     & 1     & 1\\
                            \end{tabular}
                        \end{table}
                    Nota-se que será necessário realizar a soma, logo o número será $\boxed{211_{(10)}}$.

                    \item Número em complemento de 2;
                        \begin{table}[H]
                            \centering  
                            \begin{tabular}[]{c cc cc cc cc}
                                $-2^8$ & $2^7$ & $2^6$ & $2^5$ & $2^4$ & $2^3$ & $2^2$ & $2^1$ & $2^0$\\\hline
                                  -256 & 128   & 64    & 32    & 16    & 8     & 4     & 2     & 1\\
                                     0 &   1   &  1    &  0    &  1    & 0     & 0     & 1     & 1\\
                            \end{tabular}
                        \end{table}
                    Nota-se que será necessário realizar a soma, logo o número será $\boxed{211_{(10)}}$.
                \end{enumerate}
            \end{resolution}
\newpage

        \subsection{Questão 5}
            \begin{exercise}
                Considere os números abaixo para binário de 8 bits utilizando complemento de 2 e some-os. Verifique se os números são corretos e, caso contrário, indique quais \textbf{flags} devem ser ativas.
                \begin{enumerate}[noitemsep]
                    \item 63 e 17;
                    \item 27 e -38;
                    \item -44 e -28;
                    \item -102 e - 95;
                \end{enumerate}
            \end{exercise}
            \begin{resolution}
                As seguintes conversões devem ser realizadas:
                \begin{enumerate}
                    \item Conversão de 63 e 17:
                        \begin{figure}[H]
                            \centering
                            \basetenconversiontable{63}{2}
                            \basetenconversiontable{17}{2}
                        \end{figure}
                    Realizando ajuste de sinal:
                        \begin{equation*}
                            \boxed{63_{(10)} = 00111111_{(2)}}
                            \qquad
                            \boxed{17_{(10)} = 00010001_{(2)}}
                        \end{equation*}
                    Realizando soma:
                        \begin{table}[H]
                            \centering  
                            \begin{tabular}[]{cr}
                                  &  111 1110\\
                                  & 0011 1111\\
                                + & 0001 0001\\\hline
                                  & 0101 0000\\
                            \end{tabular}
                        \end{table}
                    Conclui-se que nenhuma flag deve ser levantada e o resultado será $\boxed{80_{(10)} = 0101\;0000_{(2)}}$.
\newpage

                    \item Conversão de 27 e 39:
                        \begin{figure}[H]
                            \centering
                            \basetenconversiontable{27}{2}
                            \basetenconversiontable{38}{2}
                        \end{figure}
                    Realizando ajuste de sinal:
                        \begin{equation*}
                            \boxed{ 27_{(10)} = 00011011_{(2)}}
                            \qquad
                            \boxed{-38_{(10)} = 11011010_{(2)}}
                        \end{equation*}
                    Realizando soma:
                        \begin{table}[H]
                            \centering  
                            \begin{tabular}[]{cr}
                                  &   11 0100\\
                                  & 0001 1011\\
                                + & 1101 1010\\\hline
                                  & 1111 0101\\
                            \end{tabular}
                        \end{table}
                    Conclui-se que nenhuma flag deve ser levantada e o resultado será $\boxed{-11_{(10)} = 1111\;0100_{(2)}}$.
\newpage

                    \item Conversão de 44 e 28:
                        \begin{figure}[H]
                            \centering
                            \basetenconversiontable{44}{2}
                            \basetenconversiontable{28}{2}
                        \end{figure}
                    Realizando ajuste de sinal:
                        \begin{equation*}
                            \boxed{-44_{(10)} = 11010100_{(2)}}
                            \qquad
                            \boxed{-28_{(10)} = 11100100_{(2)}}
                        \end{equation*}
                    Realizando soma:
                        \begin{table}[H]
                            \centering  
                            \begin{tabular}[]{cr}
                                  & 1000 1000\\
                                  & 1101 0100\\
                                + & 1110 0100\\\hline
                                  &11011 1000\\
                            \end{tabular}
                        \end{table}
                    Conclui-se que nenhuma flag deve ser levantada e o resultado será $\boxed{-72_{(10)} = 1011\;1000_{(2)}}$.
\newpage

                    \item Conversão de 102 e 95:
                        \begin{figure}[H]
                            \centering
                            \basetenconversiontable{102}{2}
                            \basetenconversiontable{95}{2}
                        \end{figure}
                    Realizando ajuste de sinal:
                        \begin{equation*}
                            \boxed{-102_{(10)} = 10011010_{(2)}}
                            \qquad
                            \boxed{ -95_{(10)} = 10100001_{(2)}}
                        \end{equation*}
                    Realizando soma:
                        \begin{table}[H]
                            \centering  
                            \begin{tabular}[]{cr}
                                  &         0\\
                                  & 1001 1010\\
                                + & 1010 0001\\\hline
                                  &10011 1011\\
                            \end{tabular}
                        \end{table}
                    Conclui-se que a flag de \texttt{Overflow} deve ser levantada, pois os valores dois números negativos resultaram em um positivo $\boxed{59_{(10)} = 0011\;1011_{(2)}}$.
                \end{enumerate}
            \end{resolution}
\newpage

        \subsection{Questão 6}
            \begin{exercise}
                Converta cada número decimal em código BCD8421 e de Gray com o menor número de bits possível:
                    \begin{enumerate}[noitemsep]
                        \item 28;
                        \item 71;
                        \item 145;
                    \end{enumerate}
            \end{exercise}
            \begin{resolution}
                Nota-se que será necessário da conversão intermediária para conversão de Decimal para Gray, obtendo os seguintes resultados:
                \begin{table}[H]
                    \centering
                    \begin{tabular}[]{crrrr}
                           & Decimal & Binário   & BCD8421   & Gray\\\hline
                        1. & 28      & 0001 1100 & 0010 1000 &    1 0010\\
                        2. & 71      & 0100 0111 & 0111 0001 &  110 0100\\
                        3. & 145     & 1001 0001 & 0001 0100 0101 & 1101 1001\\\hline
                    \end{tabular}
                \end{table}
            \end{resolution}
\newpage

        \subsection{Questão 7}
            \begin{exercise}
                Em qual base numérica $b$ a expressão $32_{(b)} + 4_{(b)} = 40_{(b)}$ está correta?
            \end{exercise}
            \begin{resolution}
                Nota-se que esta equação estaria correta considerando que uma base 6. Nesta configuração haveriam os números de 0 à 5, e $2+4 = 10$ nesta base.
            \end{resolution}
\newpage

        \subsection{Questão 8}
            \begin{exercise}
                Determine as portas lógicas com defeito analisando o diagrama de tempo abaixo:
                    \begin{figure}[H]
                        \centering
                        \includegraphics[height = 1.75cm]{es572_ex03_im01.png}
                    \end{figure} \noindent
            \end{exercise}
            \begin{resolution}
                Nota-se que:
                    \begin{enumerate}
                        \item \textbf{Defeituosa}, (A=0) \texttt{nand} (B=1) $\ne$ 0;

                        \item \textbf{Defeituosa}, (A=0) \texttt{nor} (B=0) $\ne$ 0;

                        \item \textbf{Não Defeituosa};

                        \item \textbf{Defeituosa}, (A=0) \texttt{nor} (B=1) $\ne$ 0;
                    \end{enumerate}
            \end{resolution}
\end{document}