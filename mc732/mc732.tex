\documentclass{article}
\usepackage{tpack}


\title{MC732 - Projeto Sistemas Computacionais}
\author{Guilherme Nunes Trofino}
\authorRA{217276}
\project{Resumo Teórico}


\begin{document}
    \maketitle
\newpage

    \tableofcontents
\newpage


\section{RISC-V Processors}
\subsection{Design Principles}
    \begin{enumerate}
        \item \textbf{Simplicity} favours regularity;
        \item \textbf{Smaller} is faster;
        \item Good design demands good \textbf{Compromises};
        \item Making the common case \textbf{Fast};
    \end{enumerate}

\subsection{Registers}
    \begin{table}[H]
        \centering\begin{tabular}{rl | l}\hline
            x0              & zero  & Constant Value 0\\
            x1              & ra    & Return Address\\
            x2              & sp    & Stack Pointer\\
            x3              & gp    & Global Pointer\\
            x4              & tp    & Thread Pointer\\
            x5-x7, x28-x31  & t0-t6 & Temporaries\\
            x8              & fp    & Frame Pointer\\
            x9,    x18-x27  & s0-s11& Saved Registers\\
            x10-x17         & a0-a7 & Function Arguments\\\hline
        \end{tabular}
        \caption{RISC-V Registers}
        \label{tab:riscvRegisers}
    \end{table}
\noindent Note que os temporaries registers podem ser manipulados livremente e não preocupa-se com "sujá-los".\\

\noindent Por outro lado, saved registers devem ser salvos e restaurados caso estes sejam utilizados pela função callee.

\subsection{Functions}
    \begin{enumerate}[rightmargin = \leftmargin]
        \item \textbf{Function Arguments:} Place parameters in registers x10 to x17;
        \item \textbf{Procedure Call:} Transfer control to procedure;
        \begin{scriptsize}
            \myStyleRISCV
            \begin{lstlisting}
    jal x1, ProcedureLabel  # jal ra, ProcedureLabel
            \end{lstlisting}
        \end{scriptsize}
        \item \textbf{Heap Store:} Acquire storage for procedure;
        \begin{scriptsize}
            \myStyleRISCV
            \begin{lstlisting}
    addi sp, sp, -16   # 16 = 4 * n, n: number of words
    sw   x20, 0(sp)
    sw   x19, 4(sp)
    sw   x18, 8(sp)
    sw   ra, 12(sp)
            \end{lstlisting}
        \end{scriptsize}
        \item \textbf{Function Execution:} Perform procedure's operations;
        \item \textbf{Heap Restore:} Place result in register for caller;
        \begin{scriptsize}
            \myStyleRISCV
            \begin{lstlisting}
    lw   ra, 12(sp)
    lw   x18, 8(sp)
    lw   x19, 4(sp)
    lw   x20, 0(sp)
    addi sp, sp, +16   # 16 = 4 * n, n: number of words
            \end{lstlisting}
        \end{scriptsize}
        \item \textbf{Procedure Return:} Return to place of call;
        \begin{scriptsize}
            \myStyleRISCV
            \begin{lstlisting}
    jalr x0, 0(x1)  # jalr zero, 0(ra)
            \end{lstlisting}
        \end{scriptsize}
    \end{enumerate}


\newpage\section{RISC-V Datapath}


\newpage\section{RISC-V Pipelining}
\paragraph{Definição}Performance de processadores RISC-V, quando mono ciclo, será determinado por seu caminho crítico: \texttt{load instruction}. Variar o período de ciclo para diferentes instruções violaria o design principle:
\begin{phrase}
    Making the common case \textbf{Fast}
\end{phrase}
\noindent Dessa forma implementa-se Pipelining para aprimorar o tempo de execução dos processadores RISC-V.\\

\noindent Isso só é possível pois a ISA, Instruction Set Architecture, do RISC-V é desenvolvida para o Pipelining. Todas suas instruções são 32-bits o que permite decodificá-las independentemente de forma direta e previsível.

\subsection{Estágios}
\paragraph{Definição}Considera-se que a cada ciclo cada instrução será analisada nos seguintes estágios:
\begin{figure}[H]
    \centering
    \includegraphics[width = 1\linewidth]{images/datapath_pipelined_1.png}
    \caption{Datapath Pipelined}
    \label{datapath_pipelined}
\end{figure}
\noindent Note que as setas em azul representa informações que vão da direita para esquerda, retornam ao longo do datapath podendo gerar Hazards.\\

\noindent Note também que, nesta implementação inicial, o resultado do \texttt{branch} será resolvido no quarto estágio do datapath.\\

\noindent Implementações mais otimizadas, mostradas posteriormente, obtêm o resultado adequado em estágios anteriores ao utilizar-se de mais hardware.

\subsection{Hazards}
\paragraph{Definição}Situações inesperadas que podem impossibilitar o início do processamento da próxima instrução no próximo ciclo sendo classificadas como:
\begin{enumerate}[rightmargin = \leftmargin]
    \item \textbf{Structure Hazards}: Recurso ocupado com a instrução anterior;
    \begin{enumerate}[rightmargin = \leftmargin]
        \item \texttt{Separate Memory}: Hardware extra é implementado para evitar conflito para acessar memória;
    \end{enumerate}

    \item \textbf{Data Hazards}: Dado dependente da instrução anterior;
    \begin{enumerate}[rightmargin = \leftmargin]
        \item \texttt{Forwarding}: Ocorre entre duas R Instructions:
        \begin{scriptsize}
            \myStyleRISCV
            \begin{lstlisting}
    add  x2, x1, x0 # R Instruction  add1
    add  x4, x3, x2 # R Instruction  add2
            \end{lstlisting}
        \end{scriptsize}
        Inicialmente considera-se que a próxima instrução, \texttt{add2}, só teria acesso ao dado em \texttt{x2} quando \texttt{add1} finalizasse o último estágio do pipeline e houvesse escrita no banco de registradores.\\

        \noindent Isso causaria 2 bubbles até que a instrução \texttt{add2} possa operar como representado abaixo:
        \begin{figure}[H]
            \centering
            \includegraphics[width = 0.8\textwidth]{images/hazard_data.png}
            \caption{Data Hazard}
            \label{hazardData}
        \end{figure}
        
        Entretanto o dado de \texttt{x2} já está calculado ao finalizar a etapa de execução. Dessa forma uma conexão entre a saída da ALU e a entrada da ALU eliminaria o atraso como representado abaixo:
        \begin{figure}[H]
            \centering
            \includegraphics[width = 0.8\textwidth]{images/hazard_data_forwarding.png}
            \caption{Data Hazard Forwarding}
            \label{hazardDataForwarding}
        \end{figure}
        Nota-se que para que isso seja realizado será necessário hardware para interpretar adequadamente os \texttt{Forwardings}. Posteriormente será mostrada sua implementação.

        \item \texttt{Load-Use}: Ocorre entre uma Load Instruction e uma R-Instruction:
        \begin{scriptsize}
            \myStyleRISCV
            \begin{lstlisting}
    ld   x1,  0(x0) # Load Instruction
    add  x3, x2, x1 # R    Instruction
            \end{lstlisting}
        \end{scriptsize}
        Apesar de \texttt{Forwarding} evitar atrasos neste caso o atraso é apenas reduzido, pois a próxima instrução, \texttt{add}, só terá acesso ao dado em \texttt{x1} quando \texttt{ld} finalizar o penúltimo estágio do pipeline e houver escrita na memória como representado abaixo:
        \begin{figure}[H]
            \centering
            \includegraphics[width = 0.75\textwidth]{images/hazard_data_load_use.png}
            \caption{Data Hazard Load-Use}
            \label{hazardDataLoadUse}
        \end{figure}
    \end{enumerate}

    \item \textbf{Control Hazards}: Execução dependente da instrução anterior;
    \begin{enumerate}[rightmargin = \leftmargin]
        \item \texttt{Stall}: Interrompimento da Execução;

        \item \texttt{Prediction}: Suposição do resultado:
        \begin{enumerate}
            \item \texttt{Static}:
            \begin{enumerate}[noitemsep, rightmargin = \leftmargin]
                \item Taken:
                \item Not Taken: Branch não executado;
            \end{enumerate}
            \item \texttt{Dynamic}:
            \begin{enumerate}[noitemsep, rightmargin = \leftmargin]
                \item 
            \end{enumerate}
        \end{enumerate}
    \end{enumerate}
\end{enumerate}

\subsection{Execução}
\paragraph{Definição}Será realizado simulações da execução de códigos para identificar como funcionará o pipeline em diferentes configurações.
\paragraph{Exemplo 1}
\begin{scriptsize}
    \myStyleRISCV
    \begin{lstlisting}
    add     x1, x2, x3
    sub     x4, x5, x6
    lw      x7, 0(x1)
    xor     x8, x1, x7
    not     x9, x7, x1
    sw      x4, 0(x8)

#   ciclo   estágio
#           IF      ID      EX      ME      WB  
#   1       add     -       -       -       -   
#   2       sub     add     -       -       -   
#   3       lw      sub     add     -       -       add e sub:  no forwarding no bubble
#   4       xor     lw      sub     add     -   
#   5       not     xor     lw      sub     add     add e lw:   no forwarding no bubble
#   6       not     xor     <>      lw      sub     lw e xor:   no forwarding one bubble data hazard
#   7       sw      not     xor     <>      lw  
#   8       -       sw      not     xor     <>  
#   9       -       -       sw      not     xor     xor e sw:   no forwarding no bubble
#  10       -       -       -       sw      not 
#  11       -       -       -       -       sw  
    \end{lstlisting}
\end{scriptsize}
Note que tanto no Ciclo 4 e 9 não é necessário \texttt{forwarding} pois o estágio de WB realiza a escrita dos registradores na primeira metade do ciclo enquanto o estágio de EX realiza a leitura dos registradores na segunda metade do ciclo.

\paragraph{Exemplo 2}Considere que \texttt{branch} é resolvido no quarto estágio, ME, que \texttt{branch prediction} é estática e \texttt{not taken}, qua há uma unidade de detecção de \texttt{hazard} e todos os \texttt{forwarding} são possíveis:
\begin{scriptsize}
    \myStyleRISCV
    \begin{lstlisting}
    main:   addi    t0, zero, 0x80      # addi0
            addi    t1, zero, X         # addi1         X=11
            add     t2, zero, zero      #       add0
    loop:   addi    t1, t1, -1          # addi2
            lw      t3, 0(t0)           #           lw
            add     t2, t2, t3          #       add1
            addi    t0, t0, 4           # addi3
            bne     t1, zero, loop      #           bne
            add     a0, t2, zero        #       add1
            add     t0, zero, zero      #       add2
            addi    t1, zero, 1         # addi4


#   ciclo   estágio
#           IF      ID      EX      ME      WB      execuções
#   1       addi0   -       -       -       -       : 1
#   2       addi1   addi0   -       -       -       : 1
#   3       add0    addi1   addi0   -       -       : 1

#   4       addi2   add0    addi1   addi0   -       : 1     #
#   5       lw      addi2   add0    addi1   addi0   : 1     #
#   6       add1    lw      addi2   add0    addi1   : 1     #   no forwarding
#   7       addi3   add1    lw      addi2   add0    : 1     #   no forwarding
#   8       addi3   add1    <>      lw      addi2   : 1     #   data hazard
#   9       bne     addi3   add1    <>      lw      : 1     #   no forwarding
#  10       add1    bne     addi3   add1    <>      : 1     #
#  11       add2    add1    bne     addi3   add1    : 1     #

#  11.1     <>      <>      <>      bne     addi3   : X-1   #   branch not taken loop (stage 4)
#  11.2     addi2   <>      <>      <>      bne     : X-1   #
#  11.3     lw      addi2   <>      <>      <>      : X-1   #
#  11.4     add1    lw      addi2   <>      <>      : X-1   #
#  11.5     addi3   add1    lw      addi2   <>      : X-1   #
#  11.6     addi3   add1    <>      lw      addi2   : X-1   #
#  11.7     bne     addi3   add1    <>      lw      : X-1   #
#  11.8     add1    bne     addi3   add1    <>      : X-1   #
#  11.9     add2    add1    bne     addi3   add1    : X-1   #

#  12       addi4   add2    add1    bne     addi3   : 1     #   branch not taken pass
#  13       -       addi4   add2    add1    bne     : 1     #
#  14       -       -       addi4   add2    add1    : 1     #
#  15       -       -       -       addi4   add2    : 1     #
#  16       -       -       -       -       addi4   : 1     #

#                                                   : 3 + 8*1 + 9*(X-1) + 5 = 9X + 7
    \end{lstlisting}
\end{scriptsize}
Note que no Ciclo 6, 7 e 9 não é necessário \texttt{forwarding} pois o estágio de WB realiza a escrita dos registradores na primeira metade do ciclo enquanto o estágio de EX realiza a leitura dos registradores na segunda metade do ciclo.\\

\noindent Note que quando o \texttt{branch} erra haverá descarte das operações nos estágios de IF, ID e EX gerando três \texttt{bubbles}. Apenas na última comparação não haverá descarte como ilustrado no ciclo 12.\\

\noindent Quando o \texttt{branch} é resolvido no estágio N a instrução correta só será buscada no estágio N+1.\\

\noindent Note que dessa forma, são necessários 106 Ciclos para que o programa seja inteiramente executado sobre as condições inicialmente propostas.\\

\noindent Caso o \texttt{branch} fosse resolvido no estágio ID 20 ciclos seriam economizados.\\

\noindent Caso \texttt{addi3} estivesse reordenada imediatamente antes de \texttt{add1} não haveria \texttt{data hazard} do Ciclo 8 e consequentemente 11 ciclos seriam economizados.\\

\subsection{Registradores}
\paragraph{Definição}Durante a execução de funções do Pipeline será necessário armazenar os dados das diferentes ações tomadas ao longo do Datapath para que as informações utilizadas sejam precisas e adequadas. Dessa forma implementam-se Registradores entre cada ciclo de execução para armazenar temporariamente dados necessários para garantir o funcionamento.

\paragraph{Representação}Dessa forma deve-se alterar o Pipeline para que memórias intermediárias sejam incluidas como demonstrado na seguinte figura:
\begin{figure}[H]
    \centering
    \includegraphics[width = 0.8\textwidth]{images/register_pipeline.png}
    \caption{Pipeline Registers}
    \label{pipelineRegisters}
\end{figure}
\noindent Durante uma etapa qualquer, na primeira metade do ciclo, haverá escrita no próximo registrador e, na segunda metade do ciclo, haverá leitura no registrador anterior.\\

\noindent Além disso, o bloco de Instruction Fetch será realizada na segunda metade do ciclo durante a etapa de Instrução Fetch.\\

\noindent Além disso, o bloco de Registers realiza leitura dos registradores durante a primeira metade do ciclo e a realiza escrita dos registradores durante a segunda metade do ciclo. Isso permite a atualização dos dados sem a necessidade de Forwarding a depender da ordenação das instruções no Pipeline.\\

\noindent Além disso, o bloco de Memória realiza suas operações apenas durante a segunda metade do ciclo.


\end{document}