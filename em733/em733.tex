\documentclass{article}
\usepackage{tpack}

\title{EM733 - Sistemas Produtivos}
\author{Guilherme Nunes Trofino}
\authorRA{217276}
\project{Resumo Teórico}

\begin{document}
    \maketitle
\newpage

    \tableofcontents
\newpage

    \section{Introdução}
        \paragraph{Definição}

    \subsection{Planejamento Estratégico}
        \paragraph{Definição}Analisar a atuação da empresa questionando quais objetivos específicos são esperados da produção e quais suas aspirações. Neste planejamento será necessário definir o objeto desejado para na sequência propor ações para que esse objetivo seja atigindo.

            \subsubsection{Implementação}
                \paragraph{Definição}Direção da empresa desenvolve uma estratégia e a mesma será implementada pela produção seguindo as seguintes etapas de amadurecimento:
                \begin{enumerate}[noitemsep]
                    \item \textbf{Neutralidade Interna:} Não comete erros, mantém internalizada e reage as mudanças do ambiente externo e interno;

                    \item \textbf{Neutralidade Externa:} Buscar a melhoria contínua, avalindo o desempenho interno em relação à organizações similares;

                    \item \textbf{Apoio Interno:} Visão clara dos objetivos estratégicos da organização;

                    \item \textbf{Apoio Externo:} Manter a superioridade por meio de vantagens baseadas em produção;
                \end{enumerate}

        \subsection{Objetivos da Produção}
            \paragraph{Definição}Toda empresa precisará analisar seu funcionamento dentro dos seguintes objetivos para julgar se está desempenhando de acordo com as diferentes etapas do \textbf{Planejamento Estratégico}.

            \subsubsection{Qualidade}
                \paragraph{Definição}Investir em treinamentos, apostando em técnicas como CEP, manutenção e Benchmarking.
                
            \subsubsection{Velocidade}
                \paragraph{Definição}Rapidez para receber produtos ou serviços, não avaliada isoladamente.

            \subsubsection{Confiabilidade}
                \paragraph{Definição}Capacidade de realizar as atividades no tempo prometido.

            \subsubsection{Flexibilidade}
                \paragraph{Definição}Capacidade de mudar de operação, seja no produto, volume ou entrega. Normalmente associado a Modularidade, estratégia para construir processos ou produtos complexos a partir de pequenos subsitemsas que podem ser desenvolvidas individualmente nas seguintes categorias:
                    \begin{enumerate}[rightmargin = \leftmargin]
                        \item \textbf{Component Swapping:} Módulos são combinados com uma plataforma básica criando variantes de produtos pertecentes à mesma família;

                        \item \textbf{Component Sharing:} Várias plataformas básicas, divindo o mesmo módulo, formam diferentes variantes de produtos pertecentes a diferentes famílias;

                        \item \textbf{Bus Modularity:}Plataforma básica pode ser trocada por um ou vários módulos, permite a variação no número e na localização dos módulos;
                    \end{enumerate}

            \subsubsection{Custo}
                \paragraph{Definição}Apoiado pelos demais objetivos como principal forma de mensurar a relação custo benefício dos demais, normalmente decisivo para decisões.

        \subsection{Melhoria da Produção}
            \paragraph{Definição} 
\end{document}