\documentclass{article}
\usepackage{tpack}

\title{EA611 - Circuitos II}
\author{Guilherme Nunes Trofino}
\authorRA{217276}
\project{Resumo Teórico}

\begin{document}
    \maketitle
\newpage

    \tableofcontents
\newpage


    \section{Introdução}
        \paragraph{Apresentação}Neste documento será descrito as informações necessárias para compreensão e solução de exercícios relacionados a disciplina \title. Note que este documento são notas realizadas por \author, em \today.

        \subsection{Transformada de Laplace}
            \paragraph{Definição}Conversão de uma equação diferencial em equação algébrica e uma convolução em multiplicação. Formalmente descrita pelas seguintes equações:

            \begin{multicols}{2}
                \raggedcolumns
                \paragraph{Forma Bilateral:}
                    \begin{equation}
                        \boxed{
                            F(s) = \mathcal{B} \{ f(t) \} := \int_{-\infty}^{+\infty} f(t) \; e^{-st} \; \text{d}t
                        }
                    \end{equation}

                \columnbreak

                \paragraph{Forma Unilateral:}
                    \begin{equation}
                        \boxed{
                            F(s) = \mathcal{L}\{ f(t) \} := \int_{0}^{+\infty} f(t) \; e^{-st} \; \text{d} t
                        }
                    \end{equation}
            \end{multicols}\noindent
            Note que a forma \texttt{Unilateral} será um caso particular da \texttt{Bilateral}. Além disso, no estudo de circuitos elétricos será conveniente a adoção do domínio dos complexos para análise. Assim $s = \sigma + \omega\text{j}$ onde $\text{j}$ será a \textbf{Unidade Imaginária}, evitando confusão com \textbf{Corrente Elétrica} causada pela notação matemática canónica.

            \paragraph{Transformações}A seguir encontram-se as principais transformações pela definição \texttt{Unilateral} necessárias:
                \begin{table}[H]
                    \centering
                    \begingroup
                    % \setlength{\tabcolsep}{5mm}
                    \renewcommand{\arraystretch}{1.25}
                    \begin{tabular}[]{lcc}
                                         & $f(t)$      & $\mathcal{L}\{ f(t) \}$\\\hline
                        Degrau Unitário  & $u(t)$      & $\frac{1}{s}$\\
                        Impulso Unitário & $\delta(t)$ & $1$\\
                                         & $t^{n}$     & $\frac{n!}{s^{n+1}}$\\
                                         & $e^{-at}$   & $\frac{1}{s+a}$\\
                                         & $te^{-at}$  & $\frac{1}{(s+a)^{2}}$\\
                                         & $\sin(at)$  & $\frac{a}{(s^2+a^2)}$\\
                                         & $\cos(at)$  & $\frac{s}{(s^2+a^2)}$\\
                        Seno Hiperbólico    & $\sinh(at)$  & $\frac{a}{(s^2-a^2)}$\\
                        Cosseno Hiperbólico & $\cosh(at)$  & $\frac{s}{(s^2-a^2)}$\\
                                         & $e^{at}\;\sin(bt)$  & $\frac{b}{(s-a)^2+b^2}$\\
                                         & $e^{at}\;\cos(bt)$  & $\frac{s-a}{(s-a)^2+b^2}$\\
                        Convolução       & $\int_{0}^{t} f(\varphi)\;g(t - \varphi) \text{d}\varphi$ & $F(s)\cdot G(s)$\\
                        Integral         & $\int_{0}^{t} f(\varphi)\;u(t - \varphi) \text{d}\varphi$ & $\frac{F(s)}{s}$\\
                        Derivada         & $\diff{f(\varphi)}{\varphi}$ & $s\cdot F(s)$\\
                        Frequência       & $e^{-at}f(t)$          & $F(s+a)$\\
                        Temporal         & $f(t-\tau)\mu(t-\tau)$ & $e^{-s\tau}F(s)$\\\hline
                    \end{tabular}
                    \endgroup
                    \caption{Tabela de Transformadas de Laplace}\label{table:Laplace}
                \end{table} \noindent
            Conside que as funções \textbf{Trigonométricas Hiperbólicas} são definidas pelas equações abaixo:
                \begin{equation}
                    \boxed{
                        \sinh(ax) = \frac{e^{ax} - e^{-ax}}{2}
                    }
                    \qquad
                    \boxed{
                        \cosh(ax) = \frac{e^{ax} + e^{-ax}}{2}
                    }
                \end{equation}

            \subsubsection{Degrau Unitário}
                \paragraph{Definição}Representação de descontinuidade unitária, normalmente utilizada para representar mudanças instantâneas em sistemas. Formalmente descrita pela seguinte equação:
                    \begin{equation}
                        \boxed{
                            u(x - a) = 
                            \begin{cases}
                                0, & x < a;\\
                                \frac{1}{2}, & x = a;\\
                                1, & x > a;\\
                            \end{cases}
                        }
                    \end{equation}

            \subsubsection{Impulso Unitário}
                \paragraph{Definição}Distribuição infinita no ponto zero e nula no restante da reta. Formalmente descrita pela seguinte equação:
                    \begin{equation}
                        \boxed{
                            \delta(x) = 
                            \begin{cases}
                                0, & x \neq 0;\\
                                \infty, & x = 0;\\
                            \end{cases}
                            }
                        \end{equation}
                Obedecendo:
                    \begin{equation*}
                        \int_{-\infty}^{+\infty} \delta(x) \; \text{d}x = 1
                        \quad\text{e}\quad
                        \boxed{
                            \int_{a}^{b} f(t) \delta(t)\;\text{d}t = 
                            \begin{cases}
                                f(0);   & \text{se } 0\in[a,b]\\
                                0;      & \text{se } 0\notin[a,b]\\
                            \end{cases}
                        }
                    \end{equation*}

            \subsubsection{Transformada da Deriva}
                \paragraph{Definição}Quando aplicada em uma derivada de ordem $n$ será necessário utilizar da recursão e integração por partes, obtendo a seguinte equação geral:
                    \begin{equation}
                        \boxed{
                            \mathcal{L}\left\{\diff[n]{f(\varphi)}{\varphi}\right\} = 
                            s^{n}\cdot F(s) - 
                            s^{n-1} \cdot f(0) - 
                            s^{n-1} \cdot f'(0) - \dots - 
                            s \cdot f^{n-2}(0) - 
                            f^{n-1}(0)
                        }
                    \end{equation}

        \subsection{Transformada de Componentes}
            \paragraph{Definição}Substituir as equações que descrevem cada componente empregado em um circuito através de seu equivalente em \textbf{Laplace} simplificará os cálculos e poderá integrar suas condições iniciais na análise. Nesta transformação o circuito resultante será puramente resistivo e obedecerá às \textbf{Leis de Kirchhoff}.

            \subsubsection{Capacitor}
                \paragraph{Definição}Genericamente considera-se a seguinte equação para descrever o comportamento do componente:
                    \begin{equation*}
                        v_{C}(t) = \frac{1}{C}\;\int_{0}^{t} i_{C}(t)\;\text{d}t + v_{C}(0)
                    \end{equation*}
                Aplica-se a \textbf{Transformada de Laplace}, obtendo a seguinte equação:
                    \begin{equation}
                        \boxed{
                            V_{C}(s) = \frac{1}{sC}\;I_{C}(s) + \frac{v_{C}(0)}{s}
                        }
                    \end{equation}

            \subsubsection{Indutor}
                \paragraph{Definição}Genericamente considera-se a seguinte equação para descrever o comportamento do componente:
                    \begin{equation*}
                        v_{L}(t) = L\;\diff{i_{L}(t)}{t}
                    \end{equation*}
                Aplica-se a \textbf{Transformada de Laplace}, obtendo a seguinte equação:
                    \begin{equation}
                        \boxed{
                            V_{L}(s) = sL\;I_{L}(s) - L\;I_{L}(0)
                        }
                    \end{equation}

            \subsubsection{Resistor}
                \paragraph{Definição}Genericamente considera-se a seguinte equação para descrever o comportamento do componente:
                    \begin{equation*}
                        v_{R}(t) = R\;i_{R}(t)
                    \end{equation*}
                Aplica-se a \textbf{Transformada de Laplace}, obtendo a seguinte equação:
                    \begin{equation}
                        \boxed{
                            V_{R}(s) = R\;I_{R}(s)
                        }
                    \end{equation}

        \subsection{Função de Rede}
            \paragraph{Definição}Simplificação dos circuitos de tal forma que análise seja facilitada pela utilização de suas entradas e de suas saídas sempre presupondo que as condições iniciais nulas obtidas pelas seguintes equações:
                \begin{equation}
                    \boxed{H(s) = \frac{I(s)}{V(s)}}
                \end{equation}
            Onde:
                \begin{enumerate}[noitemsep]
                    \item $V(s)$, \textbf{Entrada:} Tensão de Entrada;
                    \item $I(s)$, \textbf{Saída:} Saída de Corrente;
                \end{enumerate}
\end{document}