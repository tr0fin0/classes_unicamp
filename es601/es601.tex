\documentclass{article}
\usepackage{tpack}

\title{ES601 - Análise Linear de Sistemas}
\author{Guilherme Nunes Trofino}
\authorRA{217276}
\project{Resumo Teórico}

\begin{document}
    \maketitle
\newpage

    \tableofcontents
\newpage

    \section{Introdução}
        \paragraph{Apresentação}Neste documento será descrito as informações necessárias para compreensão e solução de exercícios relacionados a disciplina \thetitle . Note que este documento são notas realizadas por \theauthor , em \today.

        \subsection{Modelagem Mecânica}
            \paragraph{Definição}Modelos básicos para situações usualmente encontradas em sistemas mecânicos simples, descrevendo as equações necessárias para a descrição do movimento.
                \begin{multicols}{2}
                    \raggedcolumns
                    \subsubsection{Mola Ideal}
                        \paragraph{Definição}Dispositivo linear que apresenta uma \texttt{Constante Elástica} constante igual a $k$.\\

                        Assim, haverá uma força $\vec{F}$ exercida pela mola proporcional ao seu deslocamento $x$ com sentido oposto, de acordo com a seguinte equação:
                            \begin{equation}
                                \boxed{
                                    \vec{F} = - k\;\vec{x}
                                }
                            \end{equation}
                        Analogamente, no caso \textbf{Rotacional} um torque $\vec{T}$ causa um deslocamento angular $\theta$. Assim, a seguinte equação será válida:
                            \begin{equation}
                                \boxed{
                                    \vec{T} = - k\;\vec{\theta}
                                }
                            \end{equation}

                    \columnbreak

                    \subsubsection{Amortecedor Ideal}
                        \paragraph{Definição}Dispositivo linear que apresenta uma \texttt{Constante de Amortecimento} constante igual a $c$.\\

                        Assim, haverá uma força $\vec{F}$ exercida pelo amortecedor  proporcional a sua velocidade $\dot{x}$ com sentido oposto, de acordo com a seguinte equação:
                            \begin{equation}
                                \boxed{
                                    \vec{F} = - c\;\vec{\dot{x}}
                                }
                            \end{equation}
                        Analogamente, no caso \textbf{Rotacional} um torque $\vec{T}$ causa um velocidade angular $\dot{\theta}$. Assim, a seguinte equação será válida:
                            \begin{equation}
                                \boxed{
                                    \vec{T} = - k\;\vec{\dot{\theta}}
                                }
                            \end{equation}
                \end{multicols}\noindent
\newpage

\subsection{Transformada de Laplace}
    \paragraph{Definição}Conversão de uma equação diferencial em equação algébrica e uma convolução em multiplicação. Formalmente descrita pelas seguintes equações:
    \begin{multicols}{2}
        \raggedcolumns
        \paragraph{Forma Bilateral:}
        \begin{equation}
            \boxed{
                F(s) = \mathcal{B} \{ f(t) \} := \int_{-\infty}^{+\infty} f(t) \; e^{-st} \; \text{d}t
            }
        \end{equation}
        \columnbreak
        \paragraph{Forma Unilateral:}
        \begin{equation}
            \boxed{
                F(s) = \mathcal{L}\{ f(t) \} := \int_{0}^{+\infty} f(t) \; e^{-st} \; \text{d} t
            }
        \end{equation}
    \end{multicols}\noindent
    Note que a forma \texttt{Unilateral} será um caso particular da \texttt{Bilateral}. Além disso, no estudo de circuitos elétricos será conveniente a adoção do domínio dos complexos para análise. Assim $s = \sigma + \omega\text{j}$ onde $\text{j}$ será a \textbf{Unidade Imaginária}, evitando confusão com \textbf{Corrente Elétrica} causada pela notação matemática canónica.

    \paragraph{Transformações}A seguir encontram-se as principais transformações pela definição \texttt{Unilateral} necessárias:
    \begin{table}[H]
        \centering\begingroup
        % \setlength{\tabcolsep}{5mm}
        \renewcommand{\arraystretch}{1.25}
        \begin{tabular}[]{lcc}
                                & $f(t)$      & $\mathcal{L}\{ f(t) \}$\\\hline
            Degrau Unitário     & $u(t)$      & $\frac{1}{s}$\\
            Impulso Unitário    & $\delta(t)$ & $1$\\
            Rampa Unitário      & $r(t)$      & $\frac{1}{s^2}$\\
                                & $t^{n}$     & $\frac{n!}{s^{n+1}}$\\
                                & $e^{-at}$   & $\frac{1}{s+a}$\\
                                & $\frac{t^{n-1}e^{-at}}{(n-1)!}$  & $\frac{1}{(s+a)^{n}}$\\
                                & $\sin(at + b)$  & $\frac{s\sin(b) + a\cos(b)}{(s^2+a^2)}$\\
                                & $\cos(at + b)$  & $\frac{s\cos(b) + a\sin(b)}{(s^2+a^2)}$\\
            Seno Hiperbólico    & $\sinh(at)$  & $\frac{a}{(s^2-a^2)}$\\
            Cosseno Hiperbólico & $\cosh(at)$  & $\frac{s}{(s^2-a^2)}$\\
                                & $e^{at}\;\sin(bt)$  & $\frac{b}{(s-a)^2+b^2}$\\
                                & $e^{at}\;\cos(bt)$  & $\frac{s-a}{(s-a)^2+b^2}$\\
            Convolução          & $\int_{0}^{t} f(\varphi)\;g(t - \varphi) \text{d}\varphi$ & $F(s)\cdot G(s)$\\
            Integral            & $\int_{0}^{t} f(\varphi)\;u(t - \varphi) \text{d}\varphi$ & $\frac{F(s)}{s}$\\
            Derivada         & $\diff{f(\varphi)}{\varphi}$ & $s\cdot F(s)$\\
            Frequência       & $e^{-at}f(t)$          & $F(s+a)$\\
            Temporal         & $f(t-\tau)\mu(t-\tau)$ & $e^{-s\tau}F(s)$\\\hline
        \end{tabular}
        \endgroup
        \caption{Tabela de Transformadas de Laplace}\label{table:Laplace}
    \end{table} \noindent
    Conside que as funções \textbf{Trigonométricas Hiperbólicas} são definidas pelas equações abaixo:
    \begin{equation}
        \boxed{
            \sinh(ax) = \frac{e^{ax} - e^{-ax}}{2}
        }
        \qquad
        \boxed{
            \cosh(ax) = \frac{e^{ax} + e^{-ax}}{2}
        }
    \end{equation}

\subsubsection{Degrau Unitário}
    \paragraph{Definição}Representação de descontinuidade unitária, normalmente utilizada para representar mudanças instantâneas em sistemas. Formalmente descrita pela seguinte equação:
    \begin{equation}
        \boxed{
            u(x - a) = 
            \begin{cases}
                0, & x < a;\\
                \frac{1}{2}, & x = a;\\
                1, & x > a;\\
            \end{cases}
        }
    \end{equation}

\subsubsection{Impulso Unitário}
    \paragraph{Definição}Distribuição infinita no ponto zero e nula no restante da reta. Formalmente descrita pela seguinte equação:
    \begin{equation}
        \boxed{
            \delta(x-a) = 
            \begin{cases}
                0, & x \neq a;\\
                \infty, & x = a;\\
            \end{cases}
            }
        \end{equation}
    Obedecendo:
    \begin{equation*}
        \int_{-\infty}^{+\infty} \delta(x) \; \text{d}x = 1
        \quad\text{e}\quad
        \boxed{
            \int_{a}^{b} f(t) \delta(t - \tau)\;\text{d}t = 
            \begin{cases}
                f(\tau);    & \text{se } \tau\in[a,b]\\
                0;          & \text{se } \tau\notin[a,b]\\
            \end{cases}
        }
    \end{equation*}
    Aplica-se o impulso unitário para se extrair uma \textbf{Amostra} do valor de uma função em um determinado ponto.

\subsubsection{Rampa Unitário}
    \paragraph{Definição}Representação de uma função linear com coeficiente angular unitário. Formalmente descrita pela seguinte equação:
    \begin{equation}
        \boxed{
            r(x - a) = 
            \begin{cases}
                0, & x < a;\\
                x, & x > a;\\
            \end{cases}
        }
    \end{equation}

\subsubsection{Transformada da Deriva}
    \paragraph{Definição}Quando aplicada em uma derivada de ordem $n$ será necessário utilizar da recursão e integração por partes, obtendo a seguinte equação geral:
    \begin{equation}
        \boxed{
            \mathcal{L}\left\{\diff[n]{f(\varphi)}{\varphi}\right\} = 
            s^{n}\cdot F(s) - 
            s^{n-1} \cdot f(0) - 
            s^{n-2} \cdot f'(0) - \dots - 
            s \cdot f^{n-2}(0) - 
            f^{n-1}(0)
        }
    \end{equation}
\newpage

\subsection{Frações Parcias}
\subsubsection{\texttt{Residue Function}}
\href{https://www.mathworks.com/help/matlab/ref/residue.html}{\texttt{residue function}}

\subsubsection{Heaviside Cover-Up}
\href{https://en.wikipedia.org/wiki/Heaviside_cover-up_method}{Heaviside Cover-Up}

\href{https://www.mathworks.com/matlabcentral/answers/55210-discrete-system-solving-with-matlab}{Sistemas Discretos}

\newpage
\subsection{Transformada $\mathcal{Z}$}
    \paragraph{Definição}Conversão de uma equação a diferenças de um sinal discreto $f(k)$. \href{https://en.wikipedia.org/wiki/Z-transform}{Formalmente} descrita pelas seguintes equações:
    \begin{multicols}{2}
        \raggedcolumns
        \paragraph{Forma Bilateral:}
        \begin{equation}
            \boxed{
                F(z) = \mathcal{B} \{ f(k) \} := \sum_{k=-\infty}^{+\infty} f(k) \; z^{-k}
            }
        \end{equation}
        \columnbreak
        \paragraph{Forma Unilateral:}
        \begin{equation}
            \boxed{
                F(z) = \mathcal{Z}\{ f(k) \} := \sum_{k=0}^{+\infty} f(k) \; z^{-k}
            }
        \end{equation}
    \end{multicols}\noindent
    Note que a forma \texttt{Unilateral} será um caso particular da \texttt{Bilateral}.

    \paragraph{Transformações}A seguir encontram-se as principais transformações pela definição \texttt{Unilateral} necessárias:
    \begin{table}[H]
        \centering\begingroup
        % \setlength{\tabcolsep}{5mm}
        \renewcommand{\arraystretch}{1.5}
        \begin{tabular}[]{lcc}
                                & $f(k)$      & $\mathcal{Z}\{ f(k) \}$\\\hline
            Degrau Unitário     & $u(k)$      & $\frac{z}{z-1}$\\
            Impulso Unitário    & $\delta(k-k_0)$ & $z^{-k_0}$\\
            Rampa Unitário      & $r(k)$      & $\frac{z}{(z-1)^2}$\\
                                & $a^{k}$     & $\frac{z}{z-a}$\\
                                & $e^{-\omega_0k}$  & $\frac{z}{z-e^{-\omega_0}}$\\
                                & $\sin(\omega_0 k)$ & $\frac{z\sin(\omega_0)}{z^2 - 2z\cos(\omega_0) + 1}$\\
                                & $\cos(\omega_0 k)$ & $\frac{z^2 - z\cos(\omega_0)}{z^2 - 2z\cos(\omega_0) + 1}$\\
                                & $a^k\;\sin(\omega_0 k)$ & $\frac{za\sin(\omega_0)}{z^2 - 2za\cos(\omega_0) + a^2}$\\
                                & $a^k\;\cos(\omega_0 k)$ & $\frac{z^2 - za\cos(\omega_0)}{z^2 - 2za\cos(\omega_0) + a^2}$\\
            Convolução          & $f_1(k) * f_2(k)$ & $F_1(z)\cdot F_2(z)$\\
            Multiplicação       & $k f(k)$ & $-z \diff{F(z)}{z}$\\
            Escalamento         & $a^{-k}f(k)$ & $F(a^{-1}z)$\\
            Tempo Reverso       & $f(-k)$ & $F(z^{-1})$\\
            Deslocamento Avançado & $f(k+n)$ &\\
                                  & $f(k+1)$ & \multicolumn{1}{l}{$z^1\cdot F(z) - z^1\cdot f(0)$}\\
                                  & $f(k+2)$ & \multicolumn{1}{l}{$z^2\cdot F(z) - z^2\cdot f(0) - z^1\cdot f(1)$}\\
                                  & $f(k+3)$ & \multicolumn{1}{l}{$z^3\cdot F(z) - z^3\cdot f(0) - z^2\cdot f(1) - z^1\cdot f(2)$}\\
            Deslocamento Atrasado & $f(k-n)$ &\\
                                  & $f(k-1)$ & \multicolumn{1}{l}{$z^{-1}\cdot F(z) - f(-1)$}\\
                                  & $f(k-2)$ & \multicolumn{1}{l}{$z^{-2}\cdot F(z) - z^{-1}\cdot f(-1) - f(-2)$}\\
                                  & $f(k-3)$ & \multicolumn{1}{l}{$z^{-3}\cdot F(z) - z^{-2}\cdot f(-1) - z^{-1}\cdot f(-2) - f(-3)$}\\\hline
        \end{tabular}
        \endgroup
        \caption{Tabela de Transformadas $\mathcal{Z}$}\label{table:Z}
    \end{table} \noindent
    Note que a expressão $\omega_0$ representa a periodicidade da função a ser transformada. Há possibilidade de que o período da função modifique sua interpretação após a transformação por coincidir com o periodo de amostragem da função.

\subsubsection{Degrau Unitário}
    \paragraph{Definição}Representação de descontinuidade unitária, normalmente utilizada para representar mudanças instantâneas em sistemas. Formalmente descrita pela seguinte equação:
    \begin{equation}
        \boxed{
            u(k - a) = 
            \begin{cases}
                0, & k < a;\\
                \frac{1}{2}, & k = a;\\
                1, & k > a;\\
            \end{cases}
        }
    \end{equation}
    Note que essa série converge para $|z| > 1$.

\subsubsection{Impulso Unitário}
    \paragraph{Definição}Distribuição infinita no ponto zero e nula no restante da reta. Formalmente descrita pela seguinte equação:
    \begin{equation}
        \boxed{
            \delta(k-a) = 
            \begin{cases}
                0, & k \neq a;\\
                1, & k = a;\\
            \end{cases}
            }
    \end{equation}

\subsubsection{Rampa Unitário}
    \paragraph{Definição}Representação de uma função linear com coeficiente angular unitário. Formalmente descrita pela seguinte equação:
    \begin{equation}
        \boxed{
            r(k - a) = 
            \begin{cases}
                0, & k < a;\\
                k, & k \ge a;\\
            \end{cases}
        }
    \end{equation}

\subsubsection{Transformada do Deslocamento Avançado}
    \paragraph{Definição}Quando aplica-se um deslocamento que causa avanço na função pela ordem $n$ será necessário utilizar da recursão , obtendo a seguinte equação geral:
    \begin{equation}
        \boxed{
            \mathcal{Z}\{ f(k+n) \} = 
            z^{n}\cdot F(z) - 
            z^{n} \cdot f(0) - 
            z^{n-1} \cdot f(1) - \dots - 
            z^{2} \cdot f(n-2) - 
            z\cdot f(n-1)
        }
    \end{equation}
    Note que caso trate-se de condições iniciais nulas o resultado se resumirá à:
    \begin{equation*}
        \mathcal{Z}\{ f(k+n) \} = 
        z^{n}\cdot F(z)
    \end{equation*}

\subsubsection{Transformada do Deslocamento Atrasado}
    \paragraph{Definição}Quando aplica-se um deslocamento que causa atraso na função pela ordem $n$ será necessário utilizar da recombinação entre os somatórios, obtendo a seguinte equação geral:
    \begin{equation}
        \boxed{
            \mathcal{Z}\{ f(k-n) \} = 
            z^{-n}\cdot F(z) + 
            z^{-n+1}\cdot f(-1) + 
            z^{-n+2}\cdot f(-2) + \dots + 
            z^{-1}\cdot f(-n+1) - 
            f(-n)
        }
    \end{equation}
    Note que caso trate-se de um sistema causal, isto é, um sistema descrito a partir de um instante inicial, os termos em função de pontos negativos serão nulos e consequentemente a equação se reduz a:
    \begin{equation*}
        \mathcal{Z}\{ f(k-n) \; u(k-n)\} = z^{-n}\cdot F(z) 
    \end{equation*}

\section{Análise de Fourrier}
    \paragraph{Definição}Conversão de uma equação periódica qualquer como um somatório de senos e cossenos. 

\subsection{Série de Fourrier}
    \paragraph{Definição}Polinômio trigonométrico obtido pela soma finita da seguinte equação:
    \begin{equation}
    \boxed{S_N = \frac{a_0}{2} + \sum_{k=1}^{N} [a_k\;\cos(k\omega_0\;t) + b_k\;\sin(k\omega_0\;t)]}
    \end{equation}
    Onde:
    \begin{align*}
        a_k &= \frac{2}{T}\int_{-T/2}^{+T/2} x(t)\;\cos(k\omega_0\;t)\;\mathrm{d}t, \text{ onde: } k = 0, 1, 2, 3,...\\
        b_k &= \frac{2}{T}\int_{-T/2}^{+T/2} x(t)\;\sin(k\omega_0\;t)\;\mathrm{d}t, \text{ onde: } k = 1, 2, 3, 4,...\\
    \end{align*}


\subsection{Transformada de Fourrier}
    \paragraph{Definição}\href{https://en.wikipedia.org/wiki/Fourier_transform}{Formalmente} descrita pelas seguintes equações:

\subsubsection{Transformada de Fourrier Discreta}
    \paragraph{Definição}Caso particular em que o sinal analisado será discretizado para que possa ser obtido numericamente como expresso pela seguinte equação:
    \begin{equation}
        \boxed{X_k = \frac{1}{N} \sum_{n=0}^{N-1} x(n)\;\text{e}^{j\;2\pi \frac{kn}{N}}}
    \end{equation}
    Note que esta função é implementada em \href{https://www.mathworks.com/help/matlab/ref/fft.html}{Matlab}. Caso deseja-se implementar o cálculo manualmente recomenda-se utilizar $N = 1024$ ou $N = 2048$.

\subsubsection{Função sinc}
    \paragraph{Definição}Função matemática comumente utilizada na análise de sinais. \href{https://en.wikipedia.org/wiki/Sinc_function}{Formalmente} definida pelas seguintes equações:
    \begin{multicols}{2}
        \raggedcolumns
        \paragraph{Não Normalizada:}
        \begin{equation}
        \boxed{\text{sinc}(x) = \frac{\sin(x)}{x}}
        \end{equation}
        \columnbreak
        \paragraph{Normalizada:}
        \begin{equation}
        \boxed{\text{sinc}(x) = \frac{\sin(\pi x)}{\pi x}}
        \end{equation}
    \end{multicols}\noindent
    Independente do caso, quando avaliada em $x = 0$ obtem-se:
    \begin{equation*}
        \text{sinc}(0) := \lim_{x\to 0}\frac{\sin(ax)}{ax} = 1 \quad\forall a\neq 0 \in\mathbb{R}
    \end{equation*}
\end{document}