\documentclass{article}
\usepackage{tpack}

\title{ES601 - Análise Linear de Sistemas}
\author{Guilherme Nunes Trofino}
\authorRA{217276}
\project{Resumo Teórico}

\begin{document}
    \maketitle
\newpage

    \tableofcontents
\newpage

    \section{Introdução}
        \paragraph{Apresentação}Neste documento será descrito as informações necessárias para compreensão e solução de exercícios relacionados a disciplina \thetitle . Note que este documento são notas realizadas por \theauthor , em \today.

        \subsection{Modelagem Mecânica}
            \paragraph{Definição}Modelos básicos para situações usualmente encontradas em sistemas mecânicos simples, descrevendo as equações necessárias para a descrição do movimento.
                \begin{multicols}{2}
                    \raggedcolumns
                    \subsubsection{Mola Ideal}
                        \paragraph{Definição}Dispositivo linear que apresenta uma \texttt{Constante Elástica} constante igual a $k$.\\

                        Assim, haverá uma força $\vec{F}$ exercida pela mola proporcional ao seu deslocamento $x$ com sentido oposto, de acordo com a seguinte equação:
                            \begin{equation}
                                \boxed{
                                    \vec{F} = - k\;\vec{x}
                                }
                            \end{equation}
                        Analogamente, no caso \textbf{Rotacional} um torque $\vec{T}$ causa um deslocamento angular $\theta$. Assim, a seguinte equação será válida:
                            \begin{equation}
                                \boxed{
                                    \vec{T} = - k\;\vec{\theta}
                                }
                            \end{equation}

                    \columnbreak

                    \subsubsection{Amortecedor Ideal}
                        \paragraph{Definição}Dispositivo linear que apresenta uma \texttt{Constante de Amortecimento} constante igual a $c$.\\

                        Assim, haverá uma força $\vec{F}$ exercida pelo amortecedor  proporcional a sua velocidade $\dot{x}$ com sentido oposto, de acordo com a seguinte equação:
                            \begin{equation}
                                \boxed{
                                    \vec{F} = - c\;\vec{\dot{x}}
                                }
                            \end{equation}
                        Analogamente, no caso \textbf{Rotacional} um torque $\vec{T}$ causa um velocidade angular $\dot{\theta}$. Assim, a seguinte equação será válida:
                            \begin{equation}
                                \boxed{
                                    \vec{T} = - k\;\vec{\dot{\theta}}
                                }
                            \end{equation}
                \end{multicols}\noindent

\end{document}