\documentclass{article}
\usepackage{tpack}


\titleformat
    {\subsection}           % Part
    [block]                 % Part Shape
    {\normalfont\Large}     % Font Size
    {}                      % Label Numbering
    {0mm}                   % Part Separation
    {}                      % Code Before
    []                      % Code After

    \titlespacing*{\subsection}{0mm}{5mm}{2.5mm}


\title{ES601 - Análise Linear de Sistemas}
\author{Guilherme Nunes Trofino}
\authorRA{217276}
\project{Atividade Teórica}


\begin{document}
    \maketitle
\newpage

    \section{Atividade Teórica}
        \paragraph{Apresentação}Resolução das questões de Análise Linear de Sistemas por Guilherme Nunes Trofino, 217276, sobre \textbf{Sistemas de Segunda Ordem} analisados por Laplace.

        \subsection{Questão 1}
            \begin{exercise}
                Considere que os seguintes sistemas apresentem as seguines equações:
                    \begin{enumerate}
                        \item \textbf{Sistema A:}
                            \begin{equation}
                                \boxed{
                                    0.1\diff{y}{t} + y = u(t)
                                }
                                \quad
                                \text{onde:}
                                \quad
                                \begin{cases}
                                    y(0) = 10 & \text{Condição Inicial}\\
                                \end{cases}
                            \end{equation}
                        \item \textbf{Sistema B:}
                            \begin{equation}
                                \boxed{
                                    \diff[2]{y}{t} + 
                                    20\diff{y}{t} + 
                                    1\text{e}^{4} y = u
                                }
                                \quad
                                \text{onde:}
                                \quad
                                \begin{cases}
                                    y(0) = 0           & \text{Condição Inicial}\\
                                    \diff{y}{t}(0) = 1 & \text{Condição Inicial}\\
                                \end{cases}
                            \end{equation}
                    \end{enumerate}
                Implemente a resposta com condições iniciais nulas em Simulink usando o \texttt{Bloco de Transferência}, exporte para o MATLAB e compare com a resposta analítica.
                \\\\
                Repita o desenvolvimento anterior considerando as condições iniciais apresentadas.
            \end{exercise}
            \begin{resolution}
                a
            \end{resolution}
\end{document}