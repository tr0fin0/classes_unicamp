\documentclass{article}
\usepackage{tpack}


\titleformat
    {\subsection}           % Part
    [block]                 % Part Shape
    {\normalfont\Large}     % Font Size
    {}                      % Label Numbering
    {0mm}                   % Part Separation
    {}                      % Code Before
    []                      % Code After

    \titlespacing*{\subsection}{0mm}{5mm}{2.5mm}


\title{ES601 - Análise Linear de Sistemas}
\author{Guilherme Nunes Trofino}
\authorRA{217276}
\project{Atividade Teórica}


\begin{document}
    \maketitle
\newpage

    \section{Atividade Teórica}
        \paragraph{Apresentação}Resolução das questões de Análise Linear de Sistemas por Guilherme Nunes Trofino, 217276, sobre \textbf{Sistemas de Segunda Ordem} analisados por Laplace.

        \subsection{Questão A}
            \begin{exercise}
                Considere que o seguinte sistema:
                    \begin{equation}
                        \boxed{
                            0.1\diff{y}{t} + y = u(t)
                        }
                        \quad
                        \text{onde:}
                        \quad
                        \begin{cases}
                            y(0) = 10 & \text{Condição Inicial}\\
                        \end{cases}
                    \end{equation}
                Implemente a resposta com condições iniciais nulas em Simulink usando o \texttt{Bloco de Transferência}, exporte para o MATLAB e compare com a resposta analítica.
                \\\\
                Repita o desenvolvimento anterior considerando as condições iniciais apresentadas.
            \end{exercise}
            \begin{resolution}
                Primeiramente será necessário rescrever a equação que descreve o sistema para que a mesma possa ser representada no Simulink:
                    \begin{align}
                        0.1\diff{y}{t} + y    &= u & \text{Simplificação de Notação\nonumber}\\
                        0.1\dot{y} + y        &= u & \text{Aplicação de Laplace}\nonumber\\
                        0.1(Ys - y_{0}) + Y   &= \frac{1}{s}\nonumber\\
                        Y(0.1s + 1) &= \frac{1}{s} + 0.1 y_{0}\nonumber\\
                        Y(s) &= \frac{1}{s(0.1s + 1)} + \frac{0.1 y_{0}}{0.1s + 1} \label{eq:simulinkYsA}\\
                        \Aboxed{Y(s) &= \frac{y_{0}s + 10}{s^2 + 10s}}
                    \end{align}
                Neste ponto, para obter-se a equação de Transferência será necessário realizar a seguinte manipulação:
                    \begin{align}
                        H(s) &= \frac{Y(s)}{X(s)} & \text{onde: $X(s) = \frac{1}{s}$, Impulso Aplicado}\nonumber\\
                        \Aboxed{H(s) &= \frac{y_{0}s^2 + 10s}{s^2 + 10s}}
                    \end{align}
\newpage
                Desta forma, a Equação será representada no Simulink com o seguinte diagrama:
                    \begin{figure}[H]
                        \centering
                        \includegraphics[height = 4cm]{es601_ex03_im01.png}
                        \caption{Representação da Simulação no Simulink}
                    \end{figure}
                Realizando a simulação com condições iniciais nulas o seguinte gráfico será obtido:
                    \begin{figure}[H]
                        \centering
                        \includegraphics[height = 7cm]{es601_ex03_im02.png}
                        \caption{Gráfico Simulação Condições Nulas no Simulink}
                    \end{figure}
                Realizando a simulação com condições iniciais não nulas o seguinte gráfico será obtido:
                    \begin{figure}[H]
                        \centering
                        \includegraphics[height = 7cm]{es601_ex03_im03.png}
                        \caption{Gráfico Simulação Condições não Nulas no Simulink}
                    \end{figure}
                Na sequência será necessário solucionar a equação analiticamente finalizando a \textbf{Transformada de Laplace}:
                    \begin{align}
                        Y(s) &= \frac{1}{s(0.1s + 1)} + \frac{0.1 y_{0}}{0.1s + 1} & \text{Frações Parciais de (\ref{eq:simulinkYsA})}\nonumber\\
                        Y(s) &= \frac{1}{s} - \frac{0.1}{0.1s + 1} + \frac{0.1y_{0}}{0.1s + 1}\nonumber\\
                        Y(s) &= \frac{1}{s} - \frac{1}{s + 10} + \frac{y_{0}}{s + 10} & \text{Aplicação Inversa de Laplace}\nonumber\\
                        y(t)         &= 1u(t) - \text{e}^{-10t} + y_{0}\text{e}^{-10t}\nonumber\\
                        \Aboxed{y(t) &= 1u(t) + (y_{0} - 1)\text{e}^{-10t}}
                    \end{align}
                Implementando a equação analiticamente com condições iniciais nulas o seguinte gráfico será obtido:
                    \begin{figure}[H]
                        \centering
                        \includegraphics[height = 7cm]{es601_ex03_im04.png}
                        \caption{Gráfico Analítica Condições nulas no Simulink}
                    \end{figure}
                Implementando a equação analiticamente com condições iniciais não nulas o seguinte gráfico será obtido:
                    \begin{figure}[H]
                        \centering
                        \includegraphics[height = 7cm]{es601_ex03_im05.png}
                        \caption{Gráfico Analítica Condições não nulas no Simulink}
                    \end{figure}
\newpage
                Equação acima será modelada em MATLAB através do seguinte algoritmo:
                    \begin{scriptsize}
                        \myOctave
                        \lstinputlisting{es601_ex03am.m}
                    \end{scriptsize}
\newpage
                Compara-se assim as soluções analíticas e simuladas com condições iniciais nulas através do seguinte gráfico:
                    \begin{figure}[H]
                        \centering
                        \includegraphics[height = 8cm]{es601_ex03_im06.png}
                        \caption{Comparação Analítica e Simulink com Condições nulas}
                    \end{figure}
                Compara-se assim as soluções analíticas e simuladas com condições iniciais não nulas através do seguinte gráfico:
                    \begin{figure}[H]
                        \centering
                        \includegraphics[height = 8cm]{es601_ex03_im07.png}
                        \caption{Comparação Analítica e Simulink com Condições não nulas}
                    \end{figure}
            \end{resolution}
\newpage

        \subsection{Questão B}
            \begin{exercise}
                Considere que o seguinte sistema:
                    \begin{equation}
                        \boxed{
                            \diff[2]{y}{t} + 
                            20\diff{y}{t} + 
                            10^{4} y = u
                        }
                        \quad
                        \text{onde:}
                        \quad
                        \begin{cases}
                            y(0) = 0           & \text{Condição Inicial}\\
                            \diff{y}{t}(0) = 1 & \text{Condição Inicial}\\
                        \end{cases}
                    \end{equation}
                Implemente a resposta com condições iniciais nulas em Simulink usando o \texttt{Bloco de Transferência}, exporte para o MATLAB e compare com a resposta analítica.
                \\\\
                Repita o desenvolvimento anterior considerando as condições iniciais apresentadas.
            \end{exercise}
            \begin{resolution}
                Primeiramente será necessário rescrever a equação que descreve o sistema para que a mesma possa ser representada no Simulink:
                    \begin{align}
                        \diff[2]{y}{t} + 20\diff{y}{t} +10^4y &= u\nonumber\\
                        \ddot{y} + 20\dot{y} + 10^4y          &= u\nonumber\\
                        (Ys^2 - sy_{0} - y'_{0}) + 20(Ys - y_{0}) + 10^4Y &= \frac{1}{s}\nonumber\\
                        Y(s^2 + 20s + 10^4) &= \frac{1}{s} + y_{0}s + 20y_{0} + y'_ {0}\nonumber\\
                        Y(s) &= \frac{1}{s(s^2 + 20s + 10^4)} + \frac{y_{0}s + 20y_{0} + y'_{0}}{(s^2 + 20s + 10^4)} \label{eq:simulinkYsB}\\
                        \Aboxed{Y(s) &= \frac{y_{0}s^2 + 20y_{0}s + y'_{0}s + 1}{s(s^2 + 20s + 10^4)}}
                    \end{align}
                Neste ponto, para obter-se a equação de Transferência será necessário realizar a seguinte manipulação:
                    \begin{align}
                        H(s) &= \frac{Y(s)}{X(s)} & \text{onde: $X(s) = \frac{1}{s}$, Impulso Aplicado}\nonumber\\
                        \Aboxed{H(s) &= \frac{y_{0}s^2 + 20y_{0}s + y'_{0}s + 1}{s^2 + 20s + 10^4}}
                    \end{align}
\newpage
                Desta forma, a Equação será representada no Simulink com o seguinte diagrama:
                    \begin{figure}[H]
                        \centering
                        \includegraphics[height = 4cm]{es601_ex03_im11.png}
                        \caption{Representação da Simulação no Simulink}
                    \end{figure}
                Realizando a simulação com condições iniciais nulas o seguinte gráfico será obtido:
                    \begin{figure}[H]
                        \centering
                        \includegraphics[height = 7cm]{es601_ex03_im12.png}
                        \caption{Gráfico Simulação Condições Nulas no Simulink}
                    \end{figure}
                Realizando a simulação com condições iniciais não nulas o seguinte gráfico será obtido:
                    \begin{figure}[H]
                        \centering
                        \includegraphics[height = 7cm]{es601_ex03_im13.png}
                        \caption{Gráfico Simulação Condições não Nulas no Simulink}
                    \end{figure}
                Na sequência será necessário solucionar a equação analiticamente finalizando a \textbf{Transformada de Laplace} com Frações Parciais de (\ref{eq:simulinkYsB}):
                    \begin{align}
                        Y(s) &= 
                                \frac{1}{s(s^2 + 20s + 10^4)} + 
                                \frac{y_{0}s + 20y_{0} + y'_{0}}{(s^2 + 20s + 10^4)}\nonumber\\[1.5mm]
                             &= \frac{A}{s} + \frac{Bs + C}{s^2 + 20s + 10^4} = 
                             \begin{cases}
                                 As^2 + Bs^2 = 0 &\rightarrow\boxed{B = -10^{-4}}\\
                                 20As + Cs   = 0 &\rightarrow\boxed{C = -20\times10^{-4}}\\
                                 10^4A       = 1 &\rightarrow\boxed{A = 10^{-4}}\\
                             \end{cases}\nonumber\\[1.5mm]
                        Y(s) &= 
                                \frac{10^{-4}}{s} - 
                                10^{-4}\frac{s+20}{s^2 + 20s + 10^4} + 
                                y_{0}  \frac{s+20}{s^2 + 20s + 10^4} + 
                                y'_{0} \frac{1}{s^2 + 20s + 10^4}\nonumber\\
                        Y(s) &= 
                                \frac{10^{-4}}{s} + 
                                (y_{0} - 10^{-4})\frac{s+20}{s^2 + 20s + 10^4} + 
                                y'_{0} \frac{1}{s^2 + 20s + 10^4}\nonumber\\
                        Y(s) &= 
                                \frac{10^{-4}}{s} + 
                                (y_{0} - 10^{-4})\frac{s+20}{(s+10)^2 + (30\sqrt{11})^2} + 
                                y'_{0} \frac{1}{(s+10)^2 + (30\sqrt{11})^2}\nonumber
                    \end{align}
                Toma-se $a = 10^{-4}$ e $b = 30\sqrt{11}$:
                    \begin{align}
                        Y(s) &= 
                                a\frac{1}{s} + 
                                (y_{0} - a)\frac{(s+10)}{(s+10)^2 + (b)^2} + 
                                \frac{10(y_{0} - a) + y'_{0}}{b}\frac{b}{(s+10)^2 + (b)^2}\nonumber\\
                        y(t) &= au(t) + (y_{0} - a)\text{e}^{-10t}\cos(bt) + \frac{10(y_{0} - a) + y'_{0}}{b}\text{e}^{-10t}\sin(bt)\nonumber\\
                        \Aboxed{y(t) &= 10^{-4}u(t) + (y_{0} - 10^{-4})\text{e}^{-10t}\cos(30\sqrt{11}t) + \frac{10(y_{0} - 10^{-4}) + y'_{0}}{30\sqrt{11}}\text{e}^{-10t}\sin(30\sqrt{11}t)}
                    \end{align}
\newpage
                Implementando a equação analiticamente com condições iniciais nulas o seguinte gráfico será obtido:
                    \begin{figure}[H]
                        \centering
                        \includegraphics[height = 7cm]{es601_ex03_im14.png}
                        \caption{Gráfico Analítica Condições nulas no Simulink}
                    \end{figure}
                Implementando a equação analiticamente com condições iniciais não nulas o seguinte gráfico será obtido:
                    \begin{figure}[H]
                        \centering
                        \includegraphics[height = 7cm]{es601_ex03_im15.png}
                        \caption{Gráfico Analítica Condições não nulas no Simulink}
                    \end{figure}
\newpage
                Equação acima será modelada em MATLAB através do seguinte algoritmo:
                    \begin{scriptsize}
                        \myOctave
                        \lstinputlisting{es601_ex03bm.m}
                    \end{scriptsize}
\newpage
                Compara-se assim as soluções analíticas e simuladas com condições iniciais nulas através do seguinte gráfico:
                    \begin{figure}[H]
                        \centering
                        \includegraphics[height = 8cm]{es601_ex03_im16.png}
                        \caption{Comparação Analítica e Simulink com Condições nulas}
                    \end{figure}
                Compara-se assim as soluções analíticas e simuladas com condições iniciais não nulas através do seguinte gráfico:
                    \begin{figure}[H]
                        \centering
                        \includegraphics[height = 8cm]{es601_ex03_im17.png}
                        \caption{Comparação Analítica e Simulink com Condições não nulas}
                    \end{figure}
            \end{resolution}
\end{document}