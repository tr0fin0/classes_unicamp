\documentclass{article}
\usepackage{tpack}


\titleformat
    {\subsection}           % Part
    [block]                 % Part Shape
    {\normalfont\Large}     % Font Size
    {}                      % Label Numbering
    {0mm}                   % Part Separation
    {}                      % Code Before
    []                      % Code After

    \titlespacing*{\subsection}{0mm}{5mm}{2.5mm}


\title{ES601 - Análise Linear de Sistemas}
\author{Guilherme Nunes Trofino}
\authorRA{217276}
\project{Atividade Teórica}


\begin{document}
    \maketitle
\newpage

    \section{Atividade Teórica}
        \paragraph{Apresentação}Resolução das questões de Análise Linear de Sistemas por Guilherme Nunes Trofino, 217276, sobre \textbf{Sistemas de Segunda Ordem}.

        \subsection{Questão 1}
            \begin{exercise}
                Considere um sistema mecânico de segunda ordem descrito pela seguinte equação:
                    \begin{equation}
                        \boxed{
                            m\diff[2]{x}{t} + 
                            c\diff{x}{t} +
                            kx = 
                            u
                        }
                        \quad
                        \text{onde:}
                        \quad
                        \begin{cases}
                            u,                   & \text{Degrau Unitário de 1N}\\
                            m = 1\text{ kg},     & \text{Massa}\\
                            k = 1000\text{ N/m}, & \text{Constante Elástica}\\
                            c = 1\text{ Ns/m},   & \text{Amortecimento}\\
                        \end{cases}
                    \end{equation}
                Simule a resposta usando o Simulink com saída para \texttt{workspace} do MATLAB. Compare a resposta simulada e a resposta analítica.

                \paragraph{Observação}Utilizar a função \texttt{array} dentro do \texttt{workspace}.
            \end{exercise}
            \begin{resolution}
                Primeiramente será necessário rescrever a equação que descreve o sistema:
                    \begin{align}
                        m\diff[2]{x}{t} + c\diff{x}{t} + kx &= u & \text{Simplificação de Notação\nonumber}\\
                        m\ddot{x} + c\dot{x} + kx           &= u & \text{Isolamento de Variável}\nonumber\\
                        \Aboxed{\ddot{x} &= \frac{1}{m}u - \frac{c}{m}\dot{x} - \frac{k}{m}x}   & \text{Equação Simplifica}\label{eq:simulink}
                    \end{align}
                Desta forma, a Equação \ref{eq:simulink} será representada no Simulink com o seguinte diagrama:
                    
            \end{resolution}
\end{document}