\documentclass{article}
\usepackage{tpack}


\titleformat
    {\subsection}           % Part
    [block]                 % Part Shape
    {\normalfont\Large}     % Font Size
    {}                      % Label Numbering
    {0mm}                   % Part Separation
    {}                      % Code Before
    []                      % Code After

    \titlespacing*{\subsection}{0mm}{5mm}{2.5mm}


\title{ES601 - Análise Linear de Sistemas}
\author{Guilherme Nunes Trofino}
\authorRA{217276}
\project{Atividade Teórica}


\begin{document}
    \maketitle
\newpage

    \section{Atividade Teórica}
        \paragraph{Apresentação}Resolução das questões de Análise Linear de Sistemas por Guilherme Nunes Trofino, 217276, sobre \textbf{Sistemas de Segunda Ordem}.

        \subsection{Questão 1}
            \begin{exercise}
                Considere um sistema mecânico de segunda ordem descrito pela seguinte equação:
                    \begin{equation}
                        \boxed{
                            m\diff[2]{x}{t} + 
                            c\diff{x}{t} +
                            kx = 
                            u
                        }
                        \quad
                        \text{onde:}
                        \quad
                        \begin{cases}
                            u,                   & \text{Degrau Unitário de 1N}\\
                            m = 1\text{ kg},     & \text{Massa}\\
                            k = 1000\text{ N/m}, & \text{Constante Elástica}\\
                            c = 1\text{ Ns/m},   & \text{Amortecimento}\\
                        \end{cases}
                    \end{equation}
                Simule a resposta usando o Simulink com saída para \texttt{workspace} do MATLAB. Compare a resposta simulada e a resposta analítica.

                \paragraph{Observação}Utilizar a função \texttt{array} dentro do \texttt{workspace}.
            \end{exercise}
            \begin{resolution}
                Primeiramente será necessário rescrever a equação que descreve o sistema para que a mesma possa ser representada no Simulink:
                    \begin{align}
                        m\diff[2]{x}{t} + c\diff{x}{t} + kx &= u & \text{Simplificação de Notação\nonumber}\\
                        m\ddot{x} + c\dot{x} + kx           &= u & \text{Isolamento de Variável}\nonumber\\
                        \Aboxed{\ddot{x} &= \frac{1}{m}u - \frac{c}{m}\dot{x} - \frac{k}{m}x}   & \text{Equação Simplifica}\label{eq:simulink}
                    \end{align}
                Desta forma, a Equação \ref{eq:simulink} será representada no Simulink com o seguinte diagrama:
                    \begin{figure}[H]
                        \centering
                        \includegraphics[height = 4cm]{es601_ex02_im01.png}
                        \caption{Representação da Simulação no Simulink}
                    \end{figure}
\newpage
                Realizando a simulação o seguinte gráfico será obtido:
                    \begin{figure}[H]
                        \centering
                        \includegraphics[height = 7cm]{es601_ex02_im02.png}
                        \caption{Gráfico da Simulação no Simulink}
                    \end{figure}
                Na sequência será necessário solucionar a equação analiticamente através da \textbf{Transformada de Laplace}:
                    \begin{align}
                        m\diff[2]{x}{t} + c\diff{x}{t} + kx &= u \nonumber\\
                        m\ddot{x} + c\dot{x} + kx           &= u \nonumber\\
                        m(s^2X - sx_{0} -\dot{x}_{0}) + c(sX - x_{0}) + kX & = \frac{1}{s}\nonumber\\
                        X(ms^2 + cs + k) &= \frac{1}{s} + mx_{0}s + m\dot{x}_{0} + cx_{0}\nonumber\\
                        \Aboxed{X &= \frac{1}{s}\frac{1}{ms^2 + cs + k} + \frac{(ms + c)x_{0}}{ms^2 + cs + k} + \frac{m\dot{x}_{0}}{ms^2 + cs + k}}
                    \end{align}
                Note que $x_{0} = \dot{x}_{0} = 0$ como demonstrado:
                    \begin{equation*}
                        m\ddot{x}_{0} + c\dot{x}_{0} + k0 = 0
                        \iff
                        \boxed{\ddot{x}_{0} = \dot{x}_{0} = 0}
                    \end{equation*}
                Desta forma apenas a \textbf{Solução Particular}, dependente do input, fará parte da continuação da resolução, pois a \textbf{Solução Homogênea}, dependente das condições inciais, será nula. Aplica-se assim frações parciais:
                    \begin{equation*}
                        \frac{1}{s}\frac{1}{(ms^2 + cs + k)} =
                        \frac{A}{s} + \frac{Bs + C}{s^2 + as + b} = 
                        \begin{cases}
                            As^2 + B^2 = 0  &\rightarrow\boxed{B = -1/b}\\
                            Aas + Cs = 0    &\rightarrow\boxed{C = -a/b}\\
                            Ab = 1          &\rightarrow\boxed{A = +1/b}\\
                        \end{cases}
                        \quad
                        \text{onde: $a = \frac{c}{m}$ e $b = \frac{k}{m}$}
                    \end{equation*}
                Simplificando:
                    \begin{align}
                        X &= \frac{1}{1000}\frac{1}{s} - \frac{1}{1000}\frac{s+1}{(s^2 + 1s + 1000)}\nonumber\\
                        X &= \frac{1}{1000}\frac{1}{s} - \frac{1}{1000}\frac{s+1}{(s + 1/2)^2 + 3999/4}\nonumber\\
                        X &= \frac{1}{1000}\frac{1}{s} - \frac{1}{1000}\frac{s+1/2}{(s + 1/2)^2 + 3999/4} - \frac{1}{2000}\frac{2}{\sqrt{3999}}\frac{\sqrt{3999}/2}{(s + 1/2)^2 + 3999/4}\\
                        \Aboxed{x(t) &= \frac{u(t)}{1000} - \frac{1}{1000}\text{e}^{-\frac{t}{2}}\cos(\frac{\sqrt{3999}}{2}t) - \frac{1}{1000\sqrt{3999}}\text{e}^{-\frac{t}{2}}\sin(\frac{\sqrt{3999}}{2}t)}
                    \end{align}
                Equação acima será modelada em MATLAB através do seguinte algoritmo:
                    \begin{scriptsize}
                        \myOctave
                        \lstinputlisting{es601_ex02m.m}
                    \end{scriptsize}
                    \begin{figure}[H]
                        \centering
                        \includegraphics[height = 8cm]{es601_ex02_im03.png}
                        \caption{Comparação Simulação e Solução Analítica}
                    \end{figure}
            \end{resolution}
\end{document}