\documentclass{article}

\usepackage[a4paper, hmargin={20mm, 20mm}, vmargin={25mm, 30mm}]{geometry}
\usepackage[utf8]{inputenc}
\usepackage[english, main=portuguese]{babel}

\usepackage[hidelinks]{hyperref}
\usepackage{bookmark}
\usepackage{cancel}
\usepackage{comment}

\usepackage{array}
\usepackage{indentfirst}
\usepackage{multicol}
\usepackage{subfiles}


\title{LA113 - Francês}
\author{Guilherme Nunes Trofino}
\date{\today}


\begin{document}
    \maketitle
\newpage

    \tableofcontents
\newpage


\section{La Phonétique}
    \subsection{Letras}
        \paragraph{}O Francês é um idioma derivado do Latim, portanto utiliza do mesmo alfabeto que o Português. Nota-se acentuadas diferenças na pronuncia de diversas letras, principalmente do «R» fortemente marcada.
            
        \begin{center}
            \begin{tabular}{c  l  c  c  l}
                \textbf{A} & pronúncia «a»         &  & \textbf{N} & pronúncia «enne»\\
                \textbf{B} & pronúncia «be»        &  & \textbf{O} & pronúncia «o»\\
                \textbf{C} & pronúncia «ce»        &  & \textbf{P} & pronúncia «pe»\\
                \textbf{D} & pronúncia «de»        &  & \textbf{Q} & pronúncia «qü»\\
                \textbf{E} & pronúncia «e» fechado &  & \textbf{R} & pronúncia «erre»\\
                \textbf{F} & pronúncia «effe»      &  & \textbf{S} & pronúncia «esse»\\
                \textbf{G} & pronúncia «ge»        &  & \textbf{T} & pronúncia «te»\\
                \textbf{H} & pronúncia «hache»     &  & \textbf{U} & pronúncia «u» fechado\\
                \textbf{I} & pronúncia «i»         &  & \textbf{V} & pronúncia «ve»\\
                \textbf{J} & pronúncia «ji»        &  & \textbf{W} & pronúncia «double ve»\\
                \textbf{K} & pronúncia «ka»        &  & \textbf{X} & pronúncia «icx»\\
                \textbf{L} & pronúncia «be»        &  & \textbf{Y} & pronúncia «i grec»\\
                \textbf{M} & pronúncia «emme»      &  & \textbf{Z} & pronúncia «zéde»\\
            \end{tabular}
        \end{center}
        
        \paragraph{}Particularmente as consoantes, em sua maioria, não são pronunciadas ao final das palavras, ou seja, pronuciam-se as palavras até sua última vogal. Notavelmente «s» no final das palavras plurais, além do «r» ao final de verbos regulares não são pronunciadas.
        
        \paragraph{}Além das letras individuais há também encontros vocálicos que geram pronúncias particulares. Os Ditongos são em sua maioria encontros entre duas vogais enquanto as Vogais Nasais são encontros entre vogais e consoantes.
        
            \begin{center}
                \begin{tabular}{c  l  c  c  l}
                    \textbf{AI}  & pronunciado «é»         &  & \textbf{AN} & pronunciado «ã»\\
                    \textbf{AU}  & pronunciado «ô» fechado &  & \textbf{EN} & pronunciado «ën»\\
                    \textbf{AUX} & pronunciado «ô» fechado &  & \textbf{IN} & pronunciado «ën»\\
                    \textbf{EAU} & pronunciado «ô» fechado &  & \textbf{ON} & pronunciado «õn»\\
                    \textbf{EU}  & pronunciado «ê» fechado &  & \textbf{UN} & pronunciado «un» nasal\\
                    \textbf{OI}  & pronunciado «ua»        &  &  & \\
                    \textbf{OU}  & pronunciado «u» fechado &  &  & \\
                    \textbf{UI}  & pronunciado «üi»        &  &  & \\
                \end{tabular}               
            \end{center}  
            
    \subsection{Les Accents}
        \paragraph{}Os acentos no Francês, diferente do Português, não só modificam a pronúncia, porém são utilizados também para distinguir palavras.
            \begin{center}
                \begin{tabular}{c  l}
                    \textbf{É} & utilizado somente com a letra «e».\\
                    \textbf{Ê} & utilizado com «a», «o», «e» e «u».\\
                    \textbf{È} & utilizado somente com as vogais «a», «e» e «u».\\
                    \textbf{Ë} & menos comum, utilizado somente «e», «o» e «i».\\                  
                \end{tabular}
            \end{center}
        
    \subsection{Vocabulaire}

\newpage

\section{Les Nombres}
    \paragraph{}Os algorismos formam os números, «Les Chinffres» e «Les Nombres» respectivamente. No francês utilizam-se números arábicos assim como o português havendo portanto semelhanças tanto na pronúncia e quanto na grafia. 

    \subsection{Nombres Ordinaux}
        \paragraph{}Os algorismos básicos, de 0 a 9, são necessários para construir os números, sendo grafados e pronunciados respectivamente como: 
            
            \begin{center}
                \begin{tabular}{r  l  c  r  l}
                    \textbf{0} & Zéro   &  & \textbf{5} & Cinq\\ 
                    \textbf{1} & Un     &  & \textbf{6} & Six\\
                    \textbf{2} & Deux   &  & \textbf{7} & Sept\\ 
                    \textbf{3} & Trois  &  & \textbf{8} & Huit\\ 
                    \textbf{4} & Quatre &  & \textbf{9} & Neuf\\ 
                \end{tabular}
            \end{center}
        
        \paragraph{}Similar ao inglês os números entre 11 e 16 possuem grafia particular enquanto entre 17 e 69 são combinações entre dezenas e unidades. A partir do 69 haverão varriações regionais que geram discussões entre os nativos.
           
            \begin{center}
                \begin{tabular}{r  l  c  r  l  c  r  l}
                    \textbf{10} & Dix      &   & \textbf{20} & Vingt        &   & \textbf{30} & Trente\\
                    \textbf{11} & Onze     &   & \textbf{21} & Vingt-et-un  &   & \textbf{31} & Trente-et-un\\
                    \textbf{12} & Douze    &   & \textbf{22} & Vingt-deux   &   & \textbf{32} & Trente-deux\\
                    \textbf{13} & Treize   &   & \textbf{23} & Vingt-trois  &   & \textbf{33} & Trente-trois\\
                    \textbf{14} & Quatorze &   & \textbf{24} & Vingt-quatre &   & \textbf{34} & Trente-quatre\\
                    \textbf{15} & Quinze   &   & \textbf{25} & Vingt-cinq   &   & \textbf{35} & Trente-cinq\\
                    \textbf{16} & Seize    &   & \textbf{26} & Vingt-six    &   & \textbf{36} & Trente-six \\
                    \textbf{17} & Dix-Sept &   & \textbf{27} & Vingt-sept   &   & \textbf{37} & Trente-sept\\
                    \textbf{18} & Dix-Huit &   & \textbf{28} & Vingt-huit   &   & \textbf{38} & Trente-huit\\
                    \textbf{19} & Dix-Neuf &   & \textbf{29} & Vingt-neuf   &   & \textbf{39} & Trente-neuf\\ [2.5ex]
                
                    \textbf{40} & Quarante        &   & \textbf{50} & Cinquante        &   & \textbf{60} & Soixante\\
                    \textbf{41} & Quarante-et-un  &   & \textbf{51} & Cinquante-et-un  &   & \textbf{61} & Soixane-et-un\\
                    \textbf{42} & Quarante-deux   &   & \textbf{52} & Cinquante-deux   &   & \textbf{62} & Soixane-deux\\
                    \textbf{43} & Quarante-trois  &   & \textbf{53} & Cinquante-trois  &   & \textbf{63} & Soixane-trois\\
                    \textbf{44} & Quarante-quatre &   & \textbf{54} & Cinquante-quatre &   & \textbf{64} & Soixane-quatre\\
                    \textbf{45} & Quarante-cinq   &   & \textbf{55} & Cinquante-cinq   &   & \textbf{65} & Soixane-cinq\\
                    \textbf{46} & Quarante-six    &   & \textbf{56} & Cinquante-six    &   & \textbf{66} & Soixane-six\\
                    \textbf{47} & Quarante-sept   &   & \textbf{57} & Cinquante-sept   &   & \textbf{67} & Soixane-sept\\
                    \textbf{48} & Quarante-huit   &   & \textbf{58} & Cinquante-huit   &   & \textbf{68} & Soixane-huit\\
                    \textbf{49} & Quarante-neuf   &   & \textbf{59} & Cinquante-neuf   &   & \textbf{69} & Soixane-neuf\\[2.5ex]
                
                    \textbf{70} & Soixante-dix      &   & \textbf{80} & Quatre-vingt        &   & \textbf{90} & Quatre-vingt-dix\\
                    \textbf{71} & Soixante-et-onze  &   & \textbf{81} & Quatre-vingt-et-un  &   & \textbf{91} & Quatre-vingt-et-onze\\
                    \textbf{72} & Soixante-douze    &   & \textbf{82} & Quatre-vingt-deux   &   & \textbf{92} & Quatre-vingt-douze\\
                    \textbf{73} & Soixante-treize   &   & \textbf{83} & Quatre-vingt-trois  &   & \textbf{93} & Quatre-vingt-treize\\
                    \textbf{74} & Soixante-quatorze &   & \textbf{84} & Quatre-vingt-quatre &   & \textbf{94} & Quatre-vingt-quatorze\\
                    \textbf{75} & Soixante-quinze   &   & \textbf{85} & Quatre-vingt-cinq   &   & \textbf{95} & Quatre-vingt-quinze\\
                    \textbf{76} & Soixante-seize    &   & \textbf{86} & Quatre-vingt-six    &   & \textbf{96} & Quatre-vingt-seize\\
                    \textbf{77} & Soixante-dix-sept &   & \textbf{87} & Quatre-vingt-sept   &   & \textbf{97} & Quatre-vingt-dix-sept\\
                    \textbf{78} & Soixante-dix-huit &   & \textbf{88} & Quatre-vingt-huit   &   & \textbf{98} & Quatre-vingt-dix-huit\\
                    \textbf{79} & Soixante-dix-neuf &   & \textbf{89} & Quatre-vingt-neuf   &   & \textbf{99} & Quatre-vingt-dix-neuf\\
                \end{tabular}
            \end{center}                 
            
        \paragraph{}Na Bélgica os números entre 70 e 99 são grafados e pronunciados como segue. Agora não se faz necessário calcular o número como na França.
            
            \begin{center}
                \begin{tabular}{r  l  c  r  l  c  r  l}               
                    \textbf{70} & Septante        &   & \textbf{80} & Octante        &   & \textbf{90} & Nonante\\
                    \textbf{71} & Septante-et-un  &   & \textbf{81} & Octante-et-un  &   & \textbf{91} & Nonante-et-un\\
                    \textbf{72} & Septante-deux   &   & \textbf{82} & Octante-deux   &   & \textbf{92} & Nonante-deux\\
                    \textbf{73} & Septante-trois  &   & \textbf{83} & Octante-trois  &   & \textbf{93} & Nonante-trois\\
                    \textbf{74} & Septante-quatre &   & \textbf{84} & Octante-quatre &   & \textbf{94} & Nonante-quatre\\
                    \textbf{75} & Septante-cinq   &   & \textbf{85} & Octante-cinq   &   & \textbf{95} & Nonante-cinq\\
                    \textbf{76} & Septante-six    &   & \textbf{86} & Octante-six    &   & \textbf{96} & Nonante-six\\
                    \textbf{77} & Septante-sept   &   & \textbf{87} & Octante-sept   &   & \textbf{97} & Nonante-sept\\
                    \textbf{78} & Septante-huit   &   & \textbf{88} & Octante-huit   &   & \textbf{98} & Nonante-huit\\
                    \textbf{79} & Septante-neuf   &   & \textbf{89} & Octante-neuf   &   & \textbf{99} & Nonante-neuf\\
                \end{tabular}
            \end{center} 
            
        \paragraph{}Na Suíça os números entre 70 e 99 são grafados e pronunciados como segue. Agora não se faz necessário calcular o número como na França.
            
            \begin{center}
                \begin{tabular}{r  l  c  r  l  c  r  l}               
                    \textbf{70} & Septante        &   & \textbf{80} & Huintante        &   & \textbf{90} & Nonante\\
                    \textbf{71} & Septante-et-un  &   & \textbf{81} & Huintante-et-un  &   & \textbf{91} & Nonante-et-u\\
                    \textbf{72} & Septante-deux   &   & \textbf{82} & Huintante-deux   &   & \textbf{92} & Nonante-deux\\
                    \textbf{73} & Septante-trois  &   & \textbf{83} & Huintante-trois  &   & \textbf{93} & Nonante-trois\\
                    \textbf{74} & Septante-quatre &   & \textbf{84} & Huintante-quatre &   & \textbf{94} & Nonante-quatre\\
                    \textbf{75} & Septante-cinq   &   & \textbf{85} & Huintante-cinq   &   & \textbf{95} & Nonante-cinq\\
                    \textbf{76} & Septante-six    &   & \textbf{86} & Huintante-six    &   & \textbf{96} & Nonante-six\\
                    \textbf{77} & Septante-sept   &   & \textbf{87} & Huintante-sept   &   & \textbf{97} & Nonante-sept\\
                    \textbf{78} & Septante-huit   &   & \textbf{88} & Huintante-huit   &   & \textbf{98} & Nonante-huit\\
                    \textbf{79} & Septante-neuf   &   & \textbf{89} & Huintante-neuf   &   & \textbf{99} & Nonante-neuf\\
                \end{tabular}
            \end{center} 
        
        \paragraph{}A partir das centenas os números são obtidos através de somas sucessivas assim como o português. A seguir estão descritos as quantias menos usais porém de sumo importância.

            \begin{center}
                \begin{tabular}{c c}
                    \textbf{100}           & Cent\\ 
                    \textbf{1.000}         & Mille\\
                    \textbf{1.000.000}     & Millons\\   
                    \textbf{1.000.000.000} & Miliard\\
                \end{tabular}
            \end{center}

    \subsection{Nombres Cardinaux}
        \paragraph{}Em Francês os números cardinais, utilizados para denotar ordem ou sequência, são obtidos a partir dos números ordinais com terminações modificadas, em linhas gerais ganham a terminação «ième» a não ser que:

            \begin{center}
                \begin{tabular}{l l}
                    \textbf{Final «q»} & retiram-se o «q» e ganham a terminação «quième».\\
                    \textbf{Final «f»} & retiram-se o «f» e ganham a terminação «vième».\\
                    \textbf{Final «e»} & retiram-se o «e» e ganham a terminação «ième».\\
                \end{tabular}
            \end{center} 

        \paragraph{}Note que nas primeiras 10 posições apenas a primeira posição é exceção as regras apresentadas anteriormente.
            
            \begin{center}
                \begin{tabular}{c l c c l}
                    \textbf{1º} & Premier et Première &  & \textbf{6º}  & Sixième\\
                    \textbf{2º} & Deuxième            &  & \textbf{7º}  & Septième\\
                    \textbf{3º} & Troisième           &  & \textbf{8º}  & Huitième\\
                    \textbf{4º} & Quatrième           &  & \textbf{9º}  & Neuvième\\
                    \textbf{5º} & Cinquième           &  & \textbf{10º} & Dixième\\
                \end{tabular}
            \end{center} 
        
        \paragraph{} Os séculos, «Siècle», e andares de construções, «Étage», são numerados com números cardinais em Francês. Além disso, a unidade de tempo segundo, «Second(e)», não segue os números cardinais, como descrito anteriormente, porém vária em gênero masculino e feminino ao contrário do português exclusivamente masculino.
            
    \subsection{Âge}
        \begin{center}
            \begin{tabular}{m{8cm} m{8cm}}
                \textbf{Français:}      Quel âge avez-vous? & \textbf{Français:}      Quel âge as-tu?\\
                \textbf{Prononciation:} «Kélage avê vu?»    & \textbf{Prononciation:} «Kélage a tu?»\\
                \textbf{Portugais:}     Qual a vossa idade? & \textbf{Portugais:}     Qual a sua idade?\\
            \end{tabular}
        \end{center}

        \begin{center}
            \begin{tabular}{l}
                \textbf{Français:}      J’ai dix-neuf ans.\\
                \textbf{Prononciation:} «Jé dix-nuf án.»\\
                \textbf{Portugais:}     Eu tenho 19 anos.\\
            \end{tabular}
        \end{center}
    
    \subsection{Les Heures}
        \paragraph{}No francês as horas são tratadas assim como no português, não há diferença entre o período da manhã e da tarde como por exemplo há no inglês. Caso deseja-se expressa que é cedo pode-se utilizar «Tôt» enquanto caso deseja-se expressar que é tarde pode-se utilizar «Tard».
            
            \begin{center}
                \begin{tabular}{m{8cm} m{8cm}}
                    \textbf{Français:}      À quelle heure vous levez-vous le matin? & \textbf{Français:}      Quelle heure est-il?\\
                    \textbf{Prononciation:}                                          & \textbf{Prononciation:} \\
                    \textbf{Portugais:}     Que horas você acorda?                   & \textbf{Portugais:}     Que horas são?\\
                \end{tabular}
            \end{center}
                
        \paragraph{}Quando deseja-se expressar alguma hora fechada, em ponto no português, pode-se utilizar «Pile». Quando deseja-se expressar horas parciais antes de uma hora cheia, x para y em português, pode-se utilizar «Moins», enquanto para expressar horas parciais posteriores a horas cheias, x e y em português, pode-se utilizar «Plus». Em qualquer dos casos, se deseja-se expressar um quarto de hora pode-se utilizar «Le Quart».
        
        \paragraph{}Os períodos da manhã e da tarde podem ser expressados respectivamente como «Du Matin» e «Du Soir». A meia dia e meia noite podem ser expressados respectivamente como «Midi» e «Minuit».

    \subsection{Vocabulaire}
            \paragraph{}Para se referir a metade de uma quantidade ou número, por exemplo, pode-se utilizar a expressão «Moitié», enquanto para se referir a meio do caminho, por exemplo, pode-se utilizar a expressão «Milieu».
            
            \paragraph{}Quando deseja-se referir ao número sucessor ou posterior pode-se utilizar a expressão «Le Chiffre d'après», enquanto quando deseja-se referir ao número antecessor ou anterior pode-se utilizar a expressão «Le Chiffres d'avant».
            
            \paragraph{}Assim como no inglês há diferentes formas para se referir a diferentes tipos de relógios. Relógios de Pulso são referidos como «Montre» enquanto Relógios de Parede são refeitos como «Horloge».
\newpage

\section{Les Pronoms}
    \paragraph{}No Francês há mais formalidade do que em português devido, em grande parte, a presença de pronomes sujeitos formais equivalentes a «Tu» e «Vosso». Além dos pronomes adequados há também códigos de conduta rigorosos ao se direcionar com estranhos, devendo sempre optar por tais formas.
    
    \paragraph{}Outro ponto a se destacar é o pronome «On», em tradução livre para português «A Gente», é particular do idioma tem seu uso restrito a situações informais. Tanto que este não possui equivalente nos pronomes tônicos.
    
    \paragraph{}Os Pronomes Toniques são utilizados em frases sem verbo, normalmente empregado em retóricas.
        
    \begin{center}
        \begin{tabular}{r l c r l}
            \textbf{Je}    & Eu    &  & \textbf{Moi}   & Meu\\
            \textbf{Tu}    & Você  &  & \textbf{Toi}   & Teu\\
            \textbf{Il}    & Ele   &  & \textbf{Lui}   & Seu\\
            \textbf{Elle}  & Ela   &  & \textbf{Elle}  & Sua\\
            \textbf{On}    & Gente &  & \textbf{-}     & -\\
            \textbf{Nous}  & Nós   &  & \textbf{Nous}  & Nosso\\
            \textbf{Vous}  & Vós   &  & \textbf{Vous}  & Vosso\\
            \textbf{Ils}   & Eles  &  & \textbf{Eux}   & Seus\\
            \textbf{Elles} & Elas  &  & \textbf{Elles} & Suas
        \end{tabular}
    \end{center}

    \paragraph{}O «Tu» em francês denota a segunda pessoa do singular, utilizado de maneira informal apenas com pessoas conhecidas ou próximas. O «Vous» em francês denota a segunda pessoa do singular de maneira formal, utilizado com pessoas desconhecidas ou superiores, além de denotar a segunda pessoa do plural. Sua conjugação se assemelha ao «You» do Inglês.
    
    \paragraph{}Os Pronomes Tônicos são comumente utilizados na língua francesa quando deseja-se evitar a repetição do sujeito ou enfatizar o que está sendo dito, recorrente no idioma como intensificador.
    
    \subsection{Pronome Possessifs}
        \paragraph{}Os Pronomes Possessivos, importantíssimos em qualquer idioma, operam diferentemente no Francês. Os pronomes concordam com o objeto ao invés do sujeito como no português. No caso do «Saº», aplicado para objetos femininos nas terceiras pessoas do singular, quando a palavra subsequente for iniciada por «Vogais» ou «H» deve-se utilizar o equivalente masculino.
        
            \begin{center}
                \begin{tabular}{c c c c}
                                   & M     & F     & P\\[1ex]
                    \textbf{Moi}   & Mon   & Ma    & Mes\\ 
                    \textbf{Toi}   & Ton   & Ta    & Tes\\  
                    \textbf{Lui}   & Son   & Sa    & Ses\\
                    \textbf{Elle}  & Son   & Sa    & Ses\\
                    \textbf{Nous}  & Notre & Notre & Nos\\
                    \textbf{Vous}  & Votre & Votre & Vos\\
                    \textbf{Eux}   & Leur  & Leur  & Leurs\\
                    \textbf{Elles} & Leur  & Leur  & Leurs\\
                \end{tabular}
            \end{center}
        
    \subsection{Articles}
        \paragraph{}Os «Articles Définis» do Francês concordam com gênero e número com as palavras que se referem, utilizado com pessoas ou objetos específicos. Quando a palavra e o artigo se iniciam com as mesmas vogais haverá abreviação, eliminando a vogal do artigo e incluindo um apóstrofo.
        
        \paragraph{}Os «Articles Indéfinis» do Francês concordam com gênero e número com as palavras que se referem, utilizando com pessoas ou objetos não específicos. Não há abreviações entre as palavras e os artigos. Note que no plural independe o gênero. 

            \begin{center}
                \begin{tabular}{r c c c c c}
                                      & MD  & FD  & & MI  & FI\\[1ex]
                    \textbf{Singular} & Le  & La  & & Un  & Une\\
                    \textbf{Plural}   & Les & Les & & Des & Des\\
                \end{tabular}
            \end{center}           

    \subsection{Réponses Objectives}
        \paragraph{}No Francês, ao contrário do Português, há respostas em concordância e discordância para perguntas tanto afirmativas quanto negativas. A tabela indica as possíveis combinações entre perguntas positivas e negativas, nas linhas, e respostas positivas e negativas, nas colunas.
            
            \begin{center}
                \begin{tabular}{c c c}
                               & P   & N\\
                    \textbf{P} & Oui & Nom\\
                    \textbf{N} & Nom & Si\\
                \end{tabular}
            \end{center}    
            
    \subsection{Préposition}
        \paragraph{Dans:}Indica principalmente «Dentro» de algo ou algum lugar. Indica «daqui quanto tempo» algo será realizado.
        
        \paragraph{En:}Indica «continência» de algo em algum lugar ou alguém. Indica «duração» de uma atividade. Utilizado juntamente com países femininos ou regiões e mês, anos e estações do ano.
        
        \paragraph{Sur:}Indica «pertecimento»
        
        \paragraph{De:}Indica «origem» de alguém ou algo.

    \subsection{Vocabulaire}
    
\newpage

\section{Les Principaux Verbes du Français}
    \paragraph{}Assim como em muitos idiomas o Francês possui verbos regulares e irregulares dos quais os mais comuns estarão conjugados nas próximas páginas.
            
    \subsection{Verbes Réguliers}
        \paragraph{}Os verbos finalizados em «er» são regulares classificados como regulares, possuindo assim a mesma conjugação. Verbos cuja antepenúltima letra for «c» terão «ç» quando conjugados com «Nous». Verbos cuja antepenúltima letra for «g» terão o «e» mantido quando conjugados com «Nous».    

        \subfile{LA113 - VR}

    \subsection{Verbes Irréguliers}
        \paragraph{}Os verbos irregulares não terão padrões claros em suas conjugações. Todavia é notável que as terceiras pessoas do singular, «Ils» e «Elles», serão distintas do restante da conjugação em todo e qualquer caso

        \subfile{LA113 - VI}

    \subsection{Vocabulaire}
        \paragraph{}Quando se utiliza o verbo «Pouvoir» juntamente com outro no infinito pode-se empregar «Puis-Je», exemplo na primeira pessoa do singular. Está é uma forma de expressar a possibilidade de realizar uma ação.
        
        \paragraph{}Se faz necessário diferenciar a aplicação dos ver «Avoir» e «Être» devido ao reduzido e distinto uso de «Possuir» no português. O verbo «Avoir» deve ser utilizado com sensações físicas como calor, frio e outros, por outro lado o verbo «Être» deve ser utilizado com sentimentos como felicidade, raiva e outros.
        
        \paragraph{}Para realizar afirmações negativas pode-se utilizar «Ne+Pas+Verbe» quando o verbo estiver no infinitivo e «Ne+Verbe+Pas» com verbos de maneira geral.

        \paragraph{}O verbo «Étudier» possui duas particularidades, quando escrito como «Étudient» quer dizer «Estudo» enquanto «Étudiant» quer dizer «Estudante».
        
        \paragraph{}A expressão «Il y a» pode significar «Tem», «Há» e ou «Existe». «Il» representa o sujeito impessoal, «y» representa o lugar, lá, e «a» representa o verbo «Avoir», ter ou possuir. Esta expressão não varia em singular  ou plural, porém há variação negativa, «Il n'y a».

        \paragraph{}A expressão «Il fault + Verbe Infinitif» denota que a ação do verbo no infinito «É necessária» ou «É precisa». Isso pode ser comparado a construção imperativa do Português. 

        \paragraph{}Os verbos «Pouvoir», «Vouloir» e «Devoir» são normalmente acompanhados de outros verbos no infinitivo expressando desejado, possibilidade e dever de algo respectivamente.
\newpage

\section{La Politesse}
    \paragraph{}Os franceses são demasiado formais em sua oratória e escrita, utilizando de diversas expressões para o tratamento cordial. Muitas dessas expressões podem ser aplicadas tanto para perguntas quanto para respostas. Para desculpar-se pode-se utilizar tanto «Pardon» quanto «Je suis désolé», sendo o primeiro mais informal e curto. Ao realizar um pedido educado recomenda-se incluir «S'il vous plaît» ao final. Após um agradecimento como «Merci» pode-se utilizar «Il n'y a pas de quoi» como resposta querem dizer «Não há de quê».
        
        \begin{center}
            \begin{tabular}{m{8cm} m{8cm}}
                \textbf{Français:}      Pardon.      & \textbf{Français:} Je suis désolé(e).\\
                \textbf{Prononciation:} Pardón.      & \textbf{Prononciation:} Je súi desóle.\\
                \textbf{Portugais:}     Desculpe-me. & \textbf{Portugais:} Sinto muito.\\[2.5ex] 

                \textbf{Français:}      S’il vous plaît. & \textbf{Français:}      Il n’y a pas de quoi.\\
                \textbf{Prononciation:} Si vu plé.       & \textbf{Prononciation:} Il in a pá de kuá.\\
                \textbf{Portugais:}     Por favor.       & \textbf{Portugais:}     Não há de que.\\[2.5ex]

                \textbf{Français:}      Merci beaucoup. & \textbf{Français:}      De rien.\\
                \textbf{Prononciation:} Merci bocú.     & \textbf{Prononciation:} De rian.\\
                \textbf{Portugais:}     Muito obrigado  & \textbf{Portugais:}     De nada.\\
            \end{tabular}
        \end{center}
    
    \subsection{Compréhension}
        \paragraph{}Diálogos com falantes nativos podem trazer dificuldades tanto pelo sotaque quanto pela velocidade. Caso trechos ou expressões idiomáticas não sejam compreendidos pode-se utilizar uma das frases abaixo para melhor o entendimento da conversa.

            \begin{center}
                \begin{tabular}{m{8cm} m{8cm}}
                    \textbf{Français:}      Excusez-moi. & \textbf{Français:}      Je n'ai pas compris.\\
                    \textbf{Prononciation:} Excuze muá.  & \textbf{Prononciation:} G né pá compri.\\
                    \textbf{Portugais:}     Desculpe-me. & \textbf{Portugais:}     Eu não entendi.\\[2.5ex]

                    \textbf{Français:}      Vous pouvez parler moins vite? & \textbf{Français:}      Vous pouvez parler puls lentement?\\
                    \textbf{Prononciation:} Vu puvê parle muan vite?       & \textbf{Prononciation:} Vu puvê parle lu lateman?\\
                    \textbf{Portugais:}     Você pode falar mais devagar?  & \textbf{Portugais:}     Você pode falar mais lentamente?\\[2.5ex] 
                    
                    \textbf{Français:}      Comment ça se pronome? & \textbf{Français:}      Qu'est-ce ça veut dire?\\
                    \textbf{Prononciation:} Commã ça se pronome?   & \textbf{Prononciation:} Kérquê sa veu dire?\\
                    \textbf{Portugais:}     O que significa?       & \textbf{Portugais:}     O que significa?\\[2.5ex]

                    \textbf{Français:}      Vous pouvez répéter? & \textbf{Français:}      Comment on dit ...?\\
                    \textbf{Prononciation:} Vu puvê repeté?      & \textbf{Prononciation:} Commã on di ...?\\
                    \textbf{Portugais:}     Você pode repetir?   & \textbf{Portugais:}     Como se diz ...?\\
                \end{tabular}
            \end{center}

    \subsection{Vocabulaire}
        \paragraph{}Para se referir a pessoas mentirosas pode-se utilizar «Menteures» ou a expressão «Ranconter des Salades».
        
        \paragraph{}Quando alguém marca um compromisso porém não aparece pode-se utilizar a expressão «Poser un Lapin» a qual equivale a «Dar um Bolo» em português.
        
        \paragraph{}Os franceses possuem respostas distintas para sucesssivos espirros, utilizadas apenas entre conhecidos como brincadeira.

            \begin{center}
                \begin{tabular}{rl}
                    \textbf{À tes souhaits:} & Saúde\\
                    \textbf{À tes amours:}   & Aos seus amores\\
                    \textbf{À tes aïeux:}    & Aos seus antepassados\\
                    \textbf{Crève!:}         & Morra\\
                \end{tabular}
            \end{center}
\newpage

\section{Les Salutations}
    \paragraph{}Os franceses tradicionais possuem fama de mau tratarem turistas e não falantes de sua língua. Comumente se ofendem quando são abordados em outros idiomas que não o Francês, por isso, sempre que possível, inicie uma conversa com Francês e posteriormente tente mudar para um idioma comum, como por exemplo o Inglês. Assim como visto com os pronomes a formalidade é essencial durante a abordagem, tanto a forma coloquial quanto a forma informal são mostradas lado a lado e devem ser utilizadas adequadamente durante uma conversa. 

        \begin{center}
            \begin{tabular}{m{8cm} m{8cm}}
                \textbf{Français:}      Bonjour / Bonsoir.   & \textbf{Français:}      Salut.\\
                \textbf{Prononciation:} «Bonjur / Bonsoar».  & \textbf{Prononciation:} «Salu.»\\
                \textbf{Portugais:}     Bom dia / Boa noite. & \textbf{Portugais:}     Oi / Olá.\\[2.5ex]
                
                \textbf{Français:}      Comment vous allez?       & \textbf{Français:}      Comment ça va?\\
                \textbf{Prononciation:} «Commã vuzavez?»          & \textbf{Prononciation:} «Commã sava?»\\
                \textbf{Portugais:}     Como o(a) senhor(a) está? & \textbf{Portugais:}     Como vai?\\[2.5ex]

                \textbf{Français:}      Vous allez bien?         & \textbf{Français:}      Ça va bien?\\
                \textbf{Prononciation:} «Vuzavez bian?»          & \textbf{Prononciation:} «Sava bian?»\\
                \textbf{Portugais:}     O(A) senhor(a) está bem? & \textbf{Portugais:}     Vai bem?\\[2.5ex]

                \textbf{Français:}      Très bien.  & \textbf{Français:}      Ça va bien.\\
                \textbf{Prononciation:} «Tre bian.» & \textbf{Prononciation:} «Sava bian.»\\
                \textbf{Portugais:}     Muito bem.  & \textbf{Portugais:}     Vou bem.\\
            \end{tabular}
        \end{center}

    \paragraph{}Os cumprimentos de um encontro também variam de acordo com a intimidade dos envolvidos, descritos a seguir em ordem decrescente de formalidade. Em caso de dúvida opte pela forma mais formal.
        
        \begin{center}
            \begin{tabular}{l}
                \textbf{Serrer la Main:}       Aperto de mãos.\\
                \textbf{S'embrasser:}          Beijo, um em cada bochecha.\\
                \textbf{Prendre dans le Bras:} Abraço.\\
            \end{tabular}
        \end{center}     
    
    \subsection{Adieu}
        \paragraph{}Ao contrário dos cumprimentos as despedidas não são categorizadas de acordo com a formalidade, podendo ser utilizadas em qualquer situação para finalizar uma conversa.

            \begin{center}
                \begin{tabular}{m{8cm} m{8cm}}
                    \textbf{Français:}      Au revoir. & \textbf{Français:}      À bientôt.\\
                    \textbf{Prononciation:} «O revua». & \textbf{Prononciation:} «A bientut».\\
                    \textbf{Portugais:}     Tchau.     & \textbf{Portugais:}     Até mais.\\    
                \end{tabular}
            \end{center}

    \subsection{Présentation Formel}
        \paragraph{}No francês há, assim como no português, ordem direta e indireta, representadas respectivamente. Na ordem indireta o uso do hífen é obrigatório, fazendo a ligação entre a pessoa e o verbo utilizado, enquanto na ordem direta nenhuma ligação é necessária.
            
            \begin{center}
                \begin{tabular}{m{8cm} m{8cm}}
                    \textbf{Français:}      Comment vous vous appelez? & \textbf{Français:}      Comment vous appelez-vous?\\
                    \textbf{Prononciation:} «Commã vu vuzapélle?»      & \textbf{Prononciation:} «Commã vuzapélle vu?»\\
                    \textbf{Portugais:}     Como tu se chamas?         & \textbf{Portugais:}     Como se chamas tu?\\     
                \end{tabular}
            \end{center}

        \paragraph{}Pronúncia-se «Vous», que acompanha «Appelez», como «Z». Note que a primeira ocorrência de «Vous» denota o pronome reflexivo enquanto a segunda acompanha a conjugação do verbo. Como há divergência no significado de cada ocorrência o hífen se faz necessário para explicitar os sentidos.

            \begin{center}
                \begin{tabular}{m{8cm} m{8cm}}
                    \textbf{Français:}      Je m’appelle Guilherme, et vous? & \textbf{Français:}      Heureux de faire votre connaissance.\\
                    \textbf{Prononciation:} «G mapéle Guilherme, e vu?»     & \textbf{Prononciation:}\\
                    \textbf{Portugais:}     Eu me chamo Guilherme e tu ?     & \textbf{Portugais:}     Feliz de conhecê-lo(a).\\[2.5ex]
                    
                    \textbf{Français:}      Ravi(e) de vous connaître! & \textbf{Français:}      Moi aussi, enchaté(e) de vous connaître!\\
                    \textbf{Prononciation:} «Rávi vu conétrre.»        & \textbf{Prononciation:} «Moa oci, anchant vu conétrre.»\\
                    \textbf{Portugais:}     Feliz em conhecê-lo(a).    & \textbf{Portugais:}     Eu também estou feliz em conhecê-lo(a).\\
                \end{tabular}
            \end{center}
    
    \subsection{Présentation Informel}
        \paragraph{}Assim como descrito anteriormente há tanto ordem direta quanto ordem indireta para a abordagem informal, reforçando que essa abordagem só deve ser utilizada com conhecidos e pessoas próximas.
        
            \begin{center}
                \begin{tabular}{m{8cm} m{8cm}}
                    \textbf{Français:}      Comment tu t’appelles? & \textbf{Français:}      Comment t’appelles-tu?\\
                    \textbf{Prononciation:} «Commã tu tapélle?»    & \textbf{Prononciation:} «Commã tapélle tu?»\\
                    \textbf{Portugais:}     Como você se chama ?   & \textbf{Portugais:}     Como se chama você?\\
                \end{tabular}
            \end{center}
    
        \paragraph{}As respostas exclusivamente informais são reduzidas visto que as formais podem ser aplicadas em uma situação informal sem problemas. Entretanto responder formalmente em uma situação em que a informalidade poderia ser usada pode gerar estranhamento ou até mesmo falta de respeito a depender da pessoa.
            
            \begin{center}
                \begin{tabular}{m{8cm} m{8cm}}
                    \textbf{Français:}      Je m’appelle Guilherme, et toi? & \textbf{Français:}      Appelle-moi Trofino.\\
                    \textbf{Prononciation:} «G mapéle Guilherme, e tua?»    & \textbf{Prononciation:} «Apéle mua Trofino»\\
                    \textbf{Portugais:}     Eu me chamo Guilherme e você ?  & \textbf{Portugais:}     Chame-me de Trofino.\\
                \end{tabular}
            \end{center}

    \subsection{Questions Présentation}
        \paragraph{}Eventualmente durante uma conversa com novas pessoas pode ser necessário perguntar o «Prénom», nome principal, o «Nom», o sobrenome, e o «Surnom», o apelido, delas. Os francês utilizam os sobrenomes para se referir as pessoas, prevalecendo mais uma vez a formalidade, por isso este será normalmente a primeira pergunta.
                 
            \begin{center}
                \begin{tabular}{l}
                    \textbf{Français:}      Quel est votre prénom ? Quel est votre nom?\\
                    \textbf{Prononciation:} \\
                    \textbf{Portugais:}     Qual é vosso nome ? Qual vosso sobrenome?\\[2.5ex]
                    
                    \textbf{Français:}      Vous pouvez épeler votre nom et prénom?\\
                    \textbf{Prononciation:} «Vu puvê eplê vótre nom e prénom?»\\
                    \textbf{Portugais:}     Você poderia soletrar vosso sobrenome e nome?\\    
                \end{tabular}
            \end{center}
    
        \paragraph{}Diferentemente do Português o emprego em Francês é tratado como ocupação da vida, sendo assim quando deseja-se conhecer mais sobre a profissão de alguém pode-se perguntar como descrito a seguir, sempre levando em conta o uso adequado da formalidade.
        
            \begin{center}
                \begin{tabular}{m{8cm} m{8cm}}
                    \textbf{Français:}      Qu’est-ce que vous faites dans la vie? & \textbf{Français:}      Qu’est-ce que tu fais dans la vie?\\
                    \textbf{Prononciation:} «Késkê vu fé dãn la vie?»              & \textbf{Prononciation:} «Késkê vu fét dãn la vie?»\\
                    \textbf{Portugais:}     Qual a vossa profissão?                & \textbf{Portugais:}     Qual a sua profissão?\\
                \end{tabular}
            \end{center}
        
        \paragraph{}Independente da pergunta a resposta será a mesma e nota-se que o «T» é fracamente pronunciado em «Etudiant».
        
            \begin{center}
                \begin{tabular}{l}
                    \textbf{Français:}      Je suis étudiant en génie de contrôle et automation.\\
                    \textbf{Prononciation:} «Je súi etudiant en réne de contrrôle t otomacion.»\\
                    \textbf{Portugais:}     Eu sou estudante de engenharia de contrôle e automação.\\ 
                \end{tabular}
            \end{center}
                
    \subsection{Vocabulaire}
        \paragraph{}Além do «Vous» pessoas desconhecidas ou superiores podem ser tratadas como «Monsieur/ Messieurs», equivalente a Senhor/Senhores para homens, e  como «Madame/Mesdames», equivalentes a Senhora/Senhoras para mulheres. Há também «Mademoiselle/Mesdemoiselles» equivalente a Senhorita/Senhoritas para mulheres novas.
        
        \paragraph{}Caso deseja-se obter as informações de contato da pessoa pergunta-se sobre «Les Coordonnées», isso já pressupõem que você deseja tais dados.

        \paragraph{}Quando deseja-se receber uma pessoa pode-se utilizar «Bienvinute» para introduzi-lá ao ambiente, sendo o equivalente de «Welcome» em inglês.

        \paragraph{}Quais são as diferenças entre appelle e appele?
\newpage

\section{Les Jours et Les Mois}
    \paragraph{}Os meses do ano se aproximam ao português pela origem latina enquanto os dias da semana são próximos ao espanhol, onde cada dia representa um astro presente no sistema solar. As estações ocorrem em períodos distintos do ano, pois a França se localiza no hemisfério norte.
        
        \raggedcolumns
        \begin{multicols}{3}
            \subsection{Les Mois}        
                \begin{tabular}{r l}
                    \textbf{Janvier}   & Janeiro\\
                    \textbf{Février}   & Fevereiro\\
                    \textbf{Mars}      & Março\\
                    \textbf{Avril}     & Abril\\
                    \textbf{Mai}       & Maio\\ 
                    \textbf{Juin}      & Junho\\
                    \textbf{Juillet}   & Julho\\
                    \textbf{Aôut}      & Agosto\\ 
                    \textbf{Septembre} & Setembro\\
                    \textbf{Octobre}   & Outubro\\
                    \textbf{Novembre}  & Novembro\\
                    \textbf{Décembre}  & Dezembro\\
                \end{tabular}    
        \columnbreak
            \subsection{Les Saisons}
                \begin{tabular}{r l}
                    \textbf{Étes}      & Verão\\
                    \textbf{Automme}   & Outono\\
                    \textbf{Hiver}     & Inverno\\
                    \textbf{Printemps} & Primavera\\
                \end{tabular}    
        \columnbreak
            \subsection{Les Jours}
                \begin{tabular}{r l}
                    \textbf{Lundi}    & Segunda-Feira\\
                    \textbf{Mardi}    & Terça-Feira\\
                    \textbf{Mercredi} & Quarta-Feira\\
                    \textbf{Jeudi}    & Quinta-Feira\\
                    \textbf{Vendredi} & Sexta-Feira\\
                    \textbf{Samedi}   & Sábado\\
                    \textbf{Dimanche} & Domingo\\ 
                \end{tabular}      
        \end{multicols}
    
    \subsection{Le Jour}
        \paragraph{}Para descrever calor pode-se utilizar a expressão «Chaud» enquanto para descrever o frio pode-se utilizar a expressão «Froid». A chuva, assim como no português, é impessoal e pode ser nomeada como «a pluie», enquanto está chovendo pode ser dito como «il pleut» e chover pode ser dito como «pleuvoir».
        
        \paragraph{}Há um falso cognato relacionado a sol e solo em francês. Sol pode ser expressado como «Soleil» enquanto solo pode ser expressado como «Sol», o que difere do português.
           
    \subsection{Vocabulaire}
        \paragraph{}Para se referir ao próximo dia, independente do momento atual, pode-se utilizar «Lendemain» enquanto ao se referir ao amanhã, considerando o momento atual, deve-se utilizar «Demain». Para se referir a ontem pode-se utilizar «Hier». Caso deseja-se referir ao depois de amanhã pode-se utilizar «Après-demain» enquanto para indicar antes de ontem pode-se utilizar «Avant-hier».
        
        \paragraph{}Uma gíria importante para muitos idiomas porém não encontrada no português é o 24H7, no inglês, e 7J7, no francês, sendo equivalente a todos os dias da semana, 24h por dia todos os dias da semana.
        
        \paragraph{}Ao se referir aos estados impessoais, como o clima, deve-se utilizar o «Il» pois este desempenha a função de pronome impessoal.
\newpage

\section{Les Couleurs}
    \paragraph{}Os franceses atribuem as cores emoções e sentimentos de tal forma que diversas descrições utilizam delas. Outra particularidade das cores são suas variações que além de combinar em número combinam também em gênero, ausente no português.
    
        \begin{center}
            \begin{tabular}{r l}
                \textbf{Beige}     & Bege\\
                \textbf{Blanc(he)} & Branco\\        
                \textbf{Bleu(e)}   & Azul\\
                \textbf{Gris(e)}   & Cinza\\    
                \textbf{Jaune}     & Amarelo\\
                \textbf{Marron}    & Marrom\\
                \textbf{Noir(e)}   & Preto\\
                \textbf{Orange}    & Laranja\\ 
                \textbf{Rage}      & Vermelho\\         
                \textbf{Rose}      & Rosa\\
                \textbf{Vert(e)}   & Verde\\
                \textbf{Videt(te)} & Roxo\\[2.5ex]

                \textbf{Argenté(e)} & Prateada\\
                \textbf{Clair(e)}   & Claro\\
                \textbf{Doré(e)}    & Dourado\\
                \textbf{Foncé(e)}   & Escuro\\ 
            \end{tabular}
        \end{center}    
    
    \paragraph{}Por exemplo; «La Vie en Rose» demonstra uma perspectiva de vida positiva, «La Vie en Bleu» demonstra uma perspectiva de vida sonhadora e «La Vie en Vert» demonstra uma perspectiva de vida desgostosa ou melancólica.
        
    \subsection{Vocabulaire}
\newpage

\section{Le Lieu}
    \paragraph{}Novamente há forma formal e informal de questionar aonde a pessoa mora. Nota-se que no francês utiliza-se o «Habiter», em português «Habitar» ou «Viver», para denotar a moradia. Não há o conceito de moradia e sim de habitação.
       
        \begin{center}
            \begin{tabular}{m{8cm} m{8cm}}
                \textbf{Français:}      Où habites-vous?      & \textbf{Français:}      Où habites-tu?\\
                \textbf{Prononciation:} «U habites vu?»       & \textbf{Prononciation:} «U habites tu?»\\
                \textbf{Portugais:}     Onde você(s) mora(m)? & \textbf{Portugais:}     Onde você mora?\\
            \end{tabular}
        \end{center}

        \paragraph{}Independente da forma aplicada na pergunta a resposta será a mesma.
    
            \begin{center}
                \begin{tabular}{l}
                    \textbf{Français:}      J’habite à Valinhos.\\
                    \textbf{Prononciation:} «Jabbite à Valinhos.»\\
                    \textbf{Portugais:}     Eu moro em Valinhos.\\
                \end{tabular}
            \end{center}

        \paragraph{}A pronúncia de «J’habite» é semelhante a pronúncia de «Bitte» do alemão, acentuando o «T».
    
    \subsection{O Chez}
        \paragraph{}O «Chez» é utilizado para descrever locais não nomeados com nomes próprios, sendo aplicado como os artigos «Na/No» do português. Quando acompanha um «Pronome Tonique» denota «na casa de» e quando acompanha uma profissão denota «ida a(o)».
    
    \subsection{Le Monde}
        \paragraph{}No francês os países, escritos como «Pays» e pronunciados como «Peí», são classificados de acordo com suas terminações, os países femininos «Pays Féminins» terminam em «e» enquanto as demais terminações serão países masculinos «Pays Masculan» com exceção de: Mexique, Mozambique, Cambodge e Zimbabwe, que apesar de terminarem em «e» são classificados como masculinos.
        
        \paragraph{}Ao se referir a países que são iniciados por vogais é necessário utilizar a partícula «en». Alguns exemplos de países em que isso se faz necessário: Irak, Israel et Afghanistan.
        
        \paragraph{}Para se referir a pessoas oriundas de diferentes países pode-se utilizar «Étrangère» para mulheres e «Étrange» para homens. Além disso pode-se utilizar «Venir» para denotar origem, ou local de onde veio.
    
    \subsection{Les Continents}
        \begin{center}
            \begin{tabular}{r l}
                \textbf{Afrique}    & África\\
                \textbf{Amérique}   & América\\
                \textbf{Antartique} & Antártica\\
                \textbf{Asie}       & Ásia\\
                \textbf{Europe}     & Europa\\
                \textbf{Océanie}    & Oceania\\
            \end{tabular}
        \end{center}

    \paragraph{}Os «Etranger» possuem diferentes «Accent» que varia de país para país. Os países femininos serão precedidos pela preposição «En» enquanto países masculinos serão precedidos pela preposição «Au». As nacionalidades femininas são derivadas das masculinas: caso a masculina termine em «ien» a feminina será acresida de «ne», caso a masculina termine em «c» a feminina será acresida de «que», e as demais nacionalidades femininas serão acresidas de «e».

        \begin{center}
            \begin{tabular}{r l c r l}
                \textbf{L'Allemagne}    & Alemanha        &  & \textbf{Allemand(e)}       & Alemã(o)\\
                \textbf{L'Angleterre}   & Inglaterra      &  & \textbf{Anglais(e)}        & Inglês(a)\\
                \textbf{L'Argentine}    & Argentina       &  & \textbf{Argentin(e)}       & Argentino(a)\\
                \textbf{L'Autriche}     & Áustria         &  & \textbf{Austrichien(ne)}   & Austriaco(a)\\
                \textbf{Le Belgique}    & Bélgica         &  & \textbf{Belge}             & Belgo(a)\\
                \textbf{Le Brésil}      & Brasil          &  & \textbf{Brésilien(ne)}     & Brasileiro(a)\\
                \textbf{Le Chili}       & Chile           &  & \textbf{Chilien(ne)}       & Chileno(a)\\
                \textbf{La Chine}       & China           &  & \textbf{Chinois(e)}        & Chinês(a)\\
                \textbf{Le Cuba}        & Cuba            &  & \textbf{Cubain(e)}         & Cubano(a)\\
                \textbf{L'Écoss}        & Escócia         &  & \textbf{Écossais(e)}       & Escocês(a)\\
                \textbf{L'Égypte}       & Egito           &  & \textbf{Égyptien(ne)}      & Egípcio(a)\\
                \textbf{L'Espagne}      & Espanha         &  & \textbf{Espagnol(e)}       & Espanhol(a)\\
                \textbf{Les États-Unis} & Estados Unidos  &  & \textbf{Américan(e)}       & Americano(a)\\
                \textbf{Le Finlande}    & Finlândia       &  & \textbf{Finlandais(e)}     & Finlandês(a)\\
                \textbf{La France}      & França          &  & \textbf{Français(e)}       & Francês(a)\\
                \textbf{La Grèce}       & Grécia          &  & \textbf{Grec(que)}         & Grego(a)\\
                \textbf{La Hollande}    & Holanda         &  & \textbf{Hollandais(e)}     & Holandês(a)\\
                \textbf{L'Inde}         & Índia           &  & \textbf{Indien(ne)}        & Indiano(a)\\
                \textbf{L'Irak}         & Iraque          &  & \textbf{Irakien(ne)}       & Iraquiano(a)\\
                \textbf{L'Irlande}      & Irlanda         &  & \textbf{Irlandais(e)}      & Irlandês(a)\\
                \textbf{L'Italie}       & Itália          &  & \textbf{Italien(ne)}       & Italino(a)\\
                \textbf{Le Japon}       & Japão           &  & \textbf{Japonais(e)}       & Japonês(a)\\
                \textbf{Le Luxembourg}  & Luxemburgo      &  & \textbf{Luxembourgeois(e)} & Luxemburgês(a)\\
                \textbf{Le Maroc}       & Marrocos        &  & \textbf{Marocain(e)}       & Marroquino(a)\\ 
                \textbf{Le Mexique}     & México          &  & \textbf{Mexicain(e)}       & Mexicano(a)\\
                \textbf{Les Pays-Bas}   & Países Baixos   &  & \textbf{Néerlandais(e)}    & Holandês(a)\\
                \textbf{Le Pérou}       & Peru            &  & \textbf{Péruvien(ne)}      & Peruano(a)\\
                \textbf{Le Portugal}    & Portugal        &  & \textbf{Portugais(e)}      & Português(a)\\
                \textbf{La Russie}      & Rússia          &  & \textbf{Russe}             & Russo(a)\\
                \textbf{Le Suède}       & Suécia          &  & \textbf{Suèdois(e)}        & Sueco(a)\\ 
                \textbf{Le Suisse}      & Suíça           &  & \textbf{Suisse}            & Suíço(a)\\
                \textbf{La Turquie}     & Turquia         &  & \textbf{Turc(que)}         & Turco(a)\\
                \textbf{L'Ukraine}      & Ucrânia         &  & \textbf{Ukranien(ne)}      & Ucraniano(a)\\
            \end{tabular}
        \end{center}

    \subsection{Vocabulaire}
        \paragraph{}O endereço «L'adresse» necessita de diversas informações, entre elas temos «Rue» ou «Avenue», representando a rua ou avenida respectivamente. Geralmente as cidades francesas se dividem em «Boulevard» representando o Bulevar, uma maneira sofisticada de se referir ao bairro, e «Arrondissement» que poderia ser considerado como equivalente de distrito.
\newpage

\section{Les Professions}
    \paragraph{}Os franceses analisam a distribuição entre homens e mulheres nas profissões afim de estabelecer se há problemas de presença e inclusão tanto masculina quanto, principalmente, feminina no mercado de trabalho. A partir de tais análises são traçados planos de ações para desconstruir preconceitos e ideais machistas na sociedade contemporânea.
    
        \begin{center}
            \begin{tabular}{r l c}
                \textbf{Acteur}      & \textbf{Actrice}      & Ator Atriz\\
                \textbf{Agriculteur} & \textbf{Agricultrice} & Agricultor(a)\\
                \textbf{Ambulancier} & \textbf{Ambulancière} & Paramédico(a)\\
                \textbf{Chanteur}    & \textbf{Chanteuse}    & Cantor(a)\\
                \textbf{Écrivain}    & \textbf{Écrivaine}    & Escritor(a)\\
                \textbf{Enseignant}  & \textbf{Enseignante}  & Professor(a)\\
                \textbf{Infirmier}   & \textbf{Infirmière}   & Enfermeiro(a)\\
                \textbf{Joueur}      & \textbf{Joueuse}      & Jogador(a)\\
                \textbf{Mèdecin}     & \textbf{Mèdecienne}   & Médico(a)\\
                \textbf{Pharmacien}  & \textbf{Pharmacienne} & Farmacêutico(a)\\
                \textbf{Policier}    & \textbf{Policière}    & Policial\\
                \textbf{Pompier}     & \textbf{Pompière}     & Bombeiro(a)\\
                \textbf{Secrétair}   & \textbf{Secrétaire}   & Secretário(a)\\
                \textbf{Serveur}     & \textbf{Serveuse}     & Garçon\\
                \textbf{Sculpteur}   & \textbf{Sculpteuse}   & Escultor(a)\\
                \textbf{Vendeur}     & \textbf{Vendeuse}     & Vendedor(a)\\
            \end{tabular}
        \end{center}    

    \paragraph{}Aqui as profissões foram dividas segundo o que pesquisas apontam ser a prevalência entre homens e mulheres em cada caso, entretanto não são regras e sim tendências. Em francês as profissões são substantivos masculinos e suas terminações são modificadas, seguindo algumas regras, para se adequar ao gênero feminino. São elas:
        
        \begin{center}
            \begin{tabular}{r l}
                Terminação \textbf{(i)er:} & Há substituição por \textbf{(i)ère}.\\
                Terminação \textbf{ien:}   & Há substituição por \textbf{ienne}.\\
                Terminação \textbf{eur:}   & Há substituição por \textbf{euse}.\\
                Terminação \textbf{teur:}  & Há substituição por \textbf{trice}.\\
            \end{tabular}
        \end{center} 
        
    \subsection{Vocabulaire}
        \paragraph{}Policias também podem ser chamados como «Gendarme», tratamento formal, e como «Flic», tratamento informal.
\newpage

\section{La Nourriture}
    \paragraph{}Os franceses possuem hábitos peculiares de alimentação, consumindo certos alimentos que são demasiados em países como o Brasil. Em linhas gerais a culinária francesa é rica repleta de pratos típicos e referência de elevada qualidade. Os alimentos podem ser divididos de acordo com o momento do dia em que são geralmente comidos.

        \begin{center}
            \begin{tabular}{r l}
                \textbf{Le Beurre}    & Manteiga\\ 
                \textbf{La Confiture} & Geleia\\ 
                \textbf{Le Ceufs}     & Ovo\\ 
                \textbf{Le Mesemen}   & Pão Sírio\\ 
                \textbf{Le Pain}      & Pão\\[2.5ex]
                
                \textbf{Le Bouteille}   & Garrafa\\
                \textbf{La Bière}       & Cerveja\\
                \textbf{Le Café}        & Café\\
                \textbf{L'Eau de Coco}  & Água de Coco\\
                \textbf{Le Maple Sirop} & Xarrope\\
                \textbf{Le Thé}         & Chá\\
                \textbf{Le Vin}         & Vinho\\[2.5ex]

                \textbf{L'Avocat}        & Abacate\\
                \textbf{La Banane}       & Banana\\
                \textbf{Le Citron}       & Limão\\
                \textbf{Le Fruit}        & Fruta\\
                \textbf{Le Pomme}        & Maça\\
                \textbf{Le Raisin}       & Uva\\
                \textbf{Le Raisins Secs} & Uva Passa\\[2.5ex]
                                         
                \textbf{La Crème Caramel} & Pudim\\
                \textbf{Le Chocolat}      & Chocolate\\
                \textbf{Le Dessert}       & Sobremesa\\
                \textbf{Le Gâteau}        & Bolo\\
                \textbf{La Tarte}         & Torta\\
                \textbf{La Tarte Tatin}   & Torta de Maça\\
                \textbf{La Vanille}       & Baunilha\\[2.5ex]
                
                \textbf{Le Ali}           & Alho\\ 
                \textbf{La Chimique}      & Fermento\\ 
                \textbf{La Faine de Blé}  & Farinha de Trigo\\ 
                \textbf{La Faine de Möis} & Farinha de Milho\\ 
                \textbf{Le Fromage}       & Queijo\\ 
                \textbf{Le Oignon}        & Cebola\\ 
                \textbf{Le Olive}         & Azeitona\\ 
                \textbf{Le Piment}        & Pimenta\\
                \textbf{La Salade}        & Salada\\
                \textbf{Le Sel}           & Sal\\ 
                \textbf{La Sucre}         & Açucar\\ 
                \textbf{La Tomate}        & Tomate\\[2.5ex] 

                \textbf{L'Aubergine}  & Berinjela\\
                \textbf{Le Barbeque}  & Churrasco\\
                \textbf{Le Cassoulet} & Feijoada Francesa\\
                \textbf{Le Couscous}  & Cuscuz\\
                \textbf{Le Foie Gras} & Fígado Gorduroso\\
                \textbf{La Fondue}    & Fondi de Queijo\\
                \textbf{La Lasagne}   & Lasanha\\
                \textbf{Le Naricot}   & Feijão\\
                \textbf{La Pâte}      & Massa\\
                \textbf{La Pizza}     & Pizza\\
                \textbf{Le Riz}       & Arroz\\
                \textbf{La Sauce}     & Molho\\
            \end{tabular}
        \end{center}

    \subsection{Exprimer Ses Goûts}
        \paragraph{}Os franceses, diferentemente dos brasileiros, possuem ampla graduação dos gostos. A seguir os mais utilizados estão descritos:
        
            \begin{center}
                \begin{tabular}{r l}
                    \textbf{Adorer}               & Adoro\\
                    \textbf{Aimer Beaucoup}       & Gosto muito\\
                    \textbf{Aimer}                & Gosto\\
                    \textbf{Ne Pas Aimer}         & Não gosto\\
                    \textbf{Ne Pas Aimer du Tout} & Não gosto de jeito nenhum\\
                    \textbf{Détester}             & Detesto\\
                \end{tabular}
            \end{center}

        \paragraph{}No francês, assim como no alemão, há possibilidade de retórica concordando ou discordando de perguntas tanto afirmativas quanto negativas. A primeira coluna indica perguntas Afirmativas, como «Você gosta de Azul?», e a segunda coluna indica perguntas Negativas, como «Você não gosta de Azul?». A primeira linha indica que você quer concordar com a pergunta e a segunda linha indica que você quer discordar da pergunta.

            \begin{center}
                \begin{tabular}{r c c}
                                       & \textbf{Affirmative} & \textbf{Négative}\\[0.5ex]
                    \textbf{Accord}    & Moi aussi            & Pas moi\\
                    \textbf{Désaccord} & Moi non plus         & Moi si
                \end{tabular}
            \end{center}
                
    \subsection{Vocabulaire}
        \paragraph{}Caso deseja-se dizer que um alimento está maduro pode-se utilizar a expressão «Mûr», enquanto quando deseja-se dizer que um alimento está fedido pode-se utilizar a expressão «Puant». «Sucre» pode ser utilizado para descrever um alimento doce.
        
        \paragraph{}Quando deseja-se um prato completo de comida pode-se utilizar a expressão «Un Plat». Ao descrever o ingrediente principal de um prato pode-se utilizar a expressão «Tbé à La». Caso deseja-se experimentar a comida pode-se utilizar a expressão «Goûter».        
        
        \paragraph{}Em restaurantes para se dirigir com ao «Graçon», cuidado, utiliza-se «Serveur». Em francês «Garçon» se refere a criança. Caso deseja-se água da torneira, pode-se pedir como «Carafe»
        
        \paragraph{}Quando alguém gosta de cozinhar pode-se utilizar a expressão «Faire la Cuisine» quando esta pessoa estiver cozinhando.

        \paragraph{}Posso expressar que uma comida esteja pronta utilizando «Prêt». Uma refeição pronto seria, por exemplo, «Prêt Repas».
\newpage

\section{Expressões Idiomáticas}
    \paragraph{}Assim como no Português o Francês possui expressões idiomáticas comuns e muito utilizadas no comunicação oral.
            
        \begin{center}
            \begin{tabular}{r l}
                \textbf{Avoir Envie}                                    & Ter Vontade\\
                \textbf{Balande le Dois}                                & Desembucha\\
                \textbf{BD Bande Dessinée}                              & HQ História em Quadrinhos\\
                \textbf{Beaucoup de Monde}                              & Muita Gente\\
                \textbf{Bof}                                            & Não é Legal\\
                \textbf{Bon Courage}                                    & Bom Trabalho\\
                \textbf{Ça Aide}                                        & Isso Ajuda\\
                \textbf{Ça m'est égal}                                  & Isso me é igual, Indiferente\\
                \textbf{Ça va, Ça vient}                                & Easy Come Easy Go\\
                \textbf{C'est Dingue}                                   & Absurdo\\
                \textbf{C'est Fou}                                      & Louco\\
                \textbf{C'est Folle}                                    & Louca\\
                \textbf{Charme}                                         & Graça / Charme\\
                \textbf{Être Prêt}                                      & Estar Pronto\\
                \textbf{D'allre te Faire Encluter}                      & Tomar no Cu\\
                \textbf{D'aller te Faire Foutre}                        & Fuder\\
                \textbf{Demi-Mot}                                       & Meias Plavras\\
                \textbf{Du Gâteau}                                      & Muito Fácil\\
                \textbf{Faire Exprès}                                   & Fazer de Prpósito\\
                \textbf{Faire la Grasse Matin}                          & Acordar Tarde\\
                \textbf{Jéclate de Rire}                                & Explodir de Rir\\
                \textbf{Je n'en Crois pas Mes Yeux}                     & Não Acredito em Meus Olhos\\
                \textbf{L'herbe est Toulours plus Verte chez le Voisin} & A Grama do Visinho sempre é mais Verde\\
                \textbf{Même pas Mal}                                   & Não me Fez Mal\\
                \textbf{Ouais}                                          & Aham\\
                \textbf{Sauve qui Peut}                                 & Salva-se Quem Puder\\
                \textbf{Toujours}                                       & Sempre\\
                \textbf{Trop}                                           & Demais\\
            \end{tabular}
        \end{center}

    \subsection{Vocabulaire}
\newpage

\section{Adjectif}
        \begin{center}
            \begin{tabular}{r l}
                \textbf{Bête}       & Burra\\
                \textbf{Blonde}     & Cabelo Loiro\\
                \textbf{Brun}       & Cabelo Escuro\\
                \textbf{Chanceuse}  & Sortudo\\
                \textbf{Cher}       & Caro\\
                \textbf{Drôle}      & Engraçado\\
                \textbf{Gentils}    & Gentil\\
                \textbf{Grand}      & Alto(a)\\
                \textbf{Gros}       & Grande\\
                \textbf{Jeun}       & Jovem\\
                \textbf{Laid}       & Feio\\
                \textbf{Luxuex}     & Luxuoso\\
                \textbf{Mauvais}    & Ruim\\
                \textbf{Mince}      & Magro(a)\\
                \textbf{Moche}      & Feio\\
                \textbf{Pauvre}     & Pobre\\
                \textbf{Prompre}    & Limpo\\
                \textbf{Riche}      & Rico(a)\\
                \textbf{Saoul}      & Bêbado\\
                \textbf{Sympatique} & Simpático\\
            \end{tabular}
        \end{center}

        \begin{center}
            \begin{tabular}{r l r l}
                \textbf{Beau}   & Bonito & \textbf{Laid}     & Feio\\
                \textbf{Calme}  & Calmo  & \textbf{Brayant}  & Barulhento\\
                \textbf{Génial} & Genial & \textbf{Nul}      & Nulo\\
                \textbf{Grand}  & Grande & \textbf{Petit}    & Pequeno\\
                \textbf{Joyeux} & Alegre & \textbf{Triste}   & Triste\\
                \textbf{Sombre} & Escuro & \textbf{Iumineux} & Claro\\
            \end{tabular}
        \end{center}

    \subsection{Vocabulaire}
        \paragraph{}Bonito pode ser dito como «Beau» que possui «Belle» como feminino que apresenta variações. Caso a palavra subsequente for uma vogal deve-se utilizar «Bel» evitando que haja um encontro entre duas vogais.

        \paragraph{}Caso deseja-se realizar um elogio simpático a alguém pode-se utilizar «Trés Mignonne».

        \paragraph{}Para expressar um valor é bom pode-se utilizar «Bon Marché».
\newpage

\section{Le Vocabulaire}
    \subsection{Posição}
        \begin{center}
            \begin{tabular}{r l}
                \textbf{À Côte de}        & Ao lado de\\
                \textbf{Au Found de}      & No Fundo\\
                \textbf{Audessus}         & Acima\\
                \textbf{Ceci}             & Esta Daqui\\
                \textbf{Cette}            & Isto\\
                \textbf{Coint}            & Canto\\
                \textbf{Contre}           & Próximo\\
                \textbf{Cóte}             & Lado\\
                \textbf{Dans}             & Dentro\\
                \textbf{Derriére}         & Átras\\
                \textbf{Devant}           & A Frente\\
                \textbf{Droite}           & Direita\\
                \textbf{En Bas}           & Embaixo\\
                \textbf{En Face de}       & Em Frente\\
                \textbf{En Haut}          & No Alto\\
                \textbf{Entre ... et ...} & Entre ... e ...\\
                \textbf{Gauche}           & Esquerda\\
                \textbf{Loin de}          & Longe\\
                \textbf{Milieu}           & Meio\\
                \textbf{Par Terre}        & No Chão\\
                \textbf{Près}             & Perto\\
                \textbf{Sous}             & Embaixo\\
                \textbf{Sur}              & Em cima\\
            \end{tabular}
        \end{center}

    \subsection{Sentimentos}
        Heurex   : Feliz\\
        Bonheur  : Felicidade\\
        Heureux  : Feliz\\
        Chaugrins: Tristeza\\
        Joie     : Alegria\\
        malheurex: infeliz\\
        colère   : raiva\\

    \subsection{Expressões Virtuais}
        k29: quoi de neuf\\
        slt: salut\\
        jms: jamais\\
        mdr: mort de rire\\
        auj: aujord'hui\\
        vs: vous\\
        ns: nous\\
        qq: quelqu'un\\
        bcp: beaucoup\\
        bjr: bonjour\\
        bsr: bonsoir\\
        dak: d'accord\\
        a2m1: à demain\\
        biz : bisous\\
        dsl : désolé(e)\\
        mat : matin\\
        p'tit déj: petit déjunner\\
        stp: s'il te plait\\
        svp: s'il vou plait\\
        appart: appartement\\
        tjs   : toujours\\
        r29   : rien de neuf\\
        Bof   : interjeição para mais ou menos\\
        

    \subsection{Advérbios}
                       Et            : E\\
                       Ou            : Ou\\
                       Mais          : Mas\\
                       Alors         : Então\\
                  En   Moyenne       : «An Muainne»: Na Média\\
                       Surtout       : Majoritariamente\\
                       Depuis        : Desde\\
                  À    Nouveau       : Novamente\\
                       Des           : Desde que\\
                       Donc          : Portanto\\
                       Surtout       : Sobretudo "Surtu"\\
                       Vraiment      : Realmente\\
                       Environ       : Aproximadamente\\
                  J    'ai l'habitude: Normalmente\\
                  Cést ça            : É isso\\
                       Ne+Verbe+que  : Seulement,    somente\\
        D'abord: Primeiramente\\
        En  Moyenne: em média\\
            ensuite: em seguida\\
            Pendant: Durante\\
            Surtout: Sobretudo\\
            environ: aproximadamente\\
            solvent: geralmente\\
        por contre : em contrapartida\\
        C'est: É\\
         1 pour identifier ou présenter une chose ou une personne\\
          c'est+déterminant+nom commum\\
          c'est+nom próprio\\
         2 pour annoncer l'arrivé d'une personne\\
          c'est+pronom tonique\\
         3 pour indiquer une date (jour/mois/année)\\
         4 pour préciser un lieu / un moment\\
         5 pour faire un commentaire sur une chose ou un opinion\\
          c'est+adjectif masculin\\
          c'est+adverbe\\
        Il/Elle est\\
         1 pour caractériser une chose ou une personne\\
          il/elle est ou ils/elles sont + adjectif (nationalité, religion, état civil, profession)\\
         2 pour indiquer l'heure a não ser que tenha um intensificador\\
         3 pour faire référence à une personne ou une chose citée précédemment\\
         4 pour exprimer une opinion impersonnelle\\
        quelqu'un: qualquer um\\
        quelque  chose      : qualquer coisa\\
                 cette      : essa / esta\\
        lors     de         : durante\\
        en       effet      : de fato\\
                 donc       : então\\
                 falloir    : é necessário\\
        le       début      : o início\\
                 églament   : igualmente\\
                 mieux      : advérbio melhor\\
        meilleur(e)(s)      : adjetivo o melhor\\
        très     vs beaucoup: très é utilizado com adjetivos, advérbios e expressões de sensações enquanto beaucoup é utilizado com verbos e substantivos utilizado para expressar quantidades grandes.
        delà     de ça      : além disso\\
        jusquià  ce         : até que\\
                 celui      : aquele\\

    \subsection{Roupas}
                                   Béret        : Boiana\\
                                   sac          : sascola/mochila\\
                                   Soutien-Gorge: Sutiã\\
                                   Robe         : Vestido\\
                          Lubettes de Vue       : Óculos de Grau\\
                          Lunettes de Soleil    : Óculos de Sol\\
                          Monture  de Lunettes  : Armação de Óculos\\
                          jupe: saia\\
                          veste: jaqueta\\
    
    \subsection{Instrumentos Musicais}
        La    Musique   : Música\\
        La    Trompette : Trompete\\
        Le    Piano     : Piano\\
        Le    Violon    : Violino\\
        La    Batterir  : Bateria\\
        La    Guitare   : Guitarra\\
        Le    Saxophone : Saxofone\\
              Qulquefois: Algumas vezes\\
        Fille Unique    : Filha única\\
        
    \subsection{Familia}
        La     Famille         : Famiê\\
        Le     Amis            : Amigos\\
        Se     Marient         : «C Marri» / Se Casar\\
               Voisins         : Vizinho\\
               Gosses          : Piralhos\\
        Petit  Ami             : Namorado\\
               Mariage         : Casamento\\
               Soeur           : Irmã\\
               Souerette       : Irmã,       gíria\\
               Demi            : Meio\\
        Grande soeur           : irmã mais velha\\
               Papa            : Papai\\
               Manan           : Mamãe\\
               Père            : Pai\\
               Mère            : Mãe\\
               Frère           : Irmão\\
               Frérot          : Irmão,      gíria\\
        Petit  Frère           : Irmão mais novo\\
        Cadet  (e)             : Caçula\\
        Demi   Soeur           : Meia irmã\\
               parents         : Pais\\
        Aîné   (e)             : Mais Velho, primogênito\\
               Divorcés        : Divorciados\\
               Mari            : Marido\\
               Beau-frère      : Cunhado\\
               Belle-soeur     : Cunhada\\
               Neveu           : Sobrinho\\
               Nièce           : Sobrinha\\
               Tonton          : Titio\\
               Tata            : Titia\\
               Oncle           : Tio\\
               Tante           : Tia\\
               Parrain         : Padrinho\\
               Marraine        : Madrinha\\
               Beau-père       : Sogro\\
               Belle-frère     : Sogra\\
               Beau-fils/gendre: Genro\\
               Belle-fille     : Nora\\
               Hommen          : Homem\\
               Époux           : Esposo\\
               Femme           : Mulher\\
               Épouse          : Esposa\\
               Enfants         : Filhos\\
               Garçon: Garoto\\
               Fille: Garota\\
               Fils            : Filho "Fiss"\\
               Fille           : Filha\\
        Petit  Fils            : Neto\\
        Petite Fille           : Neta\\
               Cousin          : Primo\\
               Cousine         : Prima\\
        Bavard(e)              : pessoa que fala de mais\\
               amoreux         : amantes/apaixonados\\
        Plus   Âge             : Velhos\\
        Âgé    (e)             : Idoso(a)\\
               Célibataire     : Solteiro\\
               Ensemble        : Juntos\\
        Copain(e)              : Namorado\\
               Couple          : Casal\\
               Jeunes          : Jovens\\
        Être Marié(e) Se Marier Le Mariage\\
        Être Séparé Se Séparer La Séparation\\
        Être Divorcé Divorcer Le Divorce\\
        Veuf : Viúvo\\
        Veuve: Viúva\\
        Marié Mariée Mariés Mariées\\
        Divorcée     Divorcés Divorcées\\
        Célibataire\\
        Veuf Veuve Veufs Veuves\\
        coup de foudre: amor a primeira vista\\
        
    
    \subsection{Animais}
                    Chat         : Gato\\
                    Chien        : Cachorro\\
                La  Souris       : A rata\\
                Le  Chave-souris : Morcego\\
                des dromadaires  : o dromedário\\
                des chameau      : camelo\\
                    grenoilles   : sapos\\
                    mouche       : mosca\\
                    coqs         : galo\\
                    renard       : raposa\\
                    lapin        : coelho\\
                    mousquitiques: mosquitos\\
                    chouette: coruja
                    animal de compagnie: animal de estimação

    
    \subsection{Frases}
                                Oú                   : Onde\\
                        Combien ça coûte?            : Quanto custa?\\
                        Quoi    de neuf?             : O que há de novo?\\
                                Qui                  : Quem\\
                        Que     Quoi                 : o que,          Quoi no final das frases\\
                                Combien              : Quanto\\
                                Où                   : Onde\\
                                Quand                : Quando\\
                                Comment              : Como\\
                                Quel                 : Qual\\
                                y                    : Pronom de lieu, não existe no português\\
                                quoi                 : né\\
                        Où      a lieu cet événement?: Onde é o evento?\\
        Quand l'évenement a lieu?: Quando e o evento?\\
        Quel   est le genre de cet événement?: Qual o género do evento?\\
        Je     peux tier au sort?            : Posso tirar a sorte? Posso chutar?\\
        Quelle est ton adresse?              : Qual seu endereço?\\

    \subsection{Material Escolar}
                    Bouteille : Garrafa «Butéiê»\\
                    Stylo     : Caneta\\
                Les Livre     : Livros\\
                Le  Smartphone: Celular\\
        L'Ordinateur: L'Ordi / Computador\\
        matrousse: estojo\\
        crayon   : lápis
        
    \subsection{Lugares}    
                                Boulangerie: Padaria\\
                                Château    : Castelo\\
                                Mairie     : Prefeitura\\
                                Gare       : Estação de Trem\\
                        Station de Métro   : Estação de Metro\\
                                Place      : Praça\\
                        une     maison     : casa\\
        Dans la rue\\
        Sur l'avenue\\
        Sur la place\\
        Mines  : Minas\\
        Bouquin: Livre\\
        supermaché: supermercado
        aéroport: aeroporto
        plage: praia
        
    \subsection{Casa}    
                                            Écrivaine    : Escritório\\
                                            abat-jour    : abajur\\
                                une         luminaire    : uma luminária\\
                                            Cuisine      : Cozinha\\
                                            Arbre        : Árvore\\
                                            tapis        : tapetes\\
                                            Portrait     : Porta Retrato\\
                                            Herbe        : Grama\\
                                            Chaîne       : Canal\\
                                            Mur          : Muro\\
                                            Studio       : Kinet\\
                                Appartement ou Appart    : Apartamento\\
                                            Manoir       : Mansão\\
                                            Château      : Castelo\\
                                            Maison       : Casa\\
                                            Pièces       : Cômodo\\
                                            Salon        : Sala\\
                                            Chambre      : Quarto\\
                                            Banleve      : Suburbio\\
                                Un          lit Simple   : Cama de Solteiro\\
                                Un          lit Double   : Cama de Casal\\
                                La          Salle de Bain: Banheiro de tomar banho\\
                                le          toilette     : banheiro comum\\
                                            jardin       : jardim\\
                                rez         de-chaussée  : térreo\\
                                salle       à manger     : sala de jantar\\
                                un          dressing     : closet\\
                                            vèranda      : jardim de inverno\\
                                lo          garage       : garange\\
                                la          cave         : porão\\
                                le          grenier      : sotão\\
        l'escalier: escada\\
        une piscine: piscina\\
        un jardin d'inver: jardim de inverno\\
        le grill  : churrasco\\
        le platond: teto\\
        le toit   : telhado\\
        l'évier torneira\\
        le lavabo: lavabo\\
        le lit   : cama\\
        l'étagère: estante\\
        la laverie         : lavanderia\\
        la table de nuit   : criado mudo\\
        le sofa            : sofá\\
        le bereau          : escrivaninha\\
        le rideau          : cortina\\
        la cheminée        : châmine\\
        la prise électrique: tomada elétrica\\
        le sol             : piso\\
        le ventilateur     : ventilador\\
        la fenêtre         : janela\\
        la porte           : porta\\
        l'air conditionnée la climatisation: ar condicionado\\
        la clé: chave\\
        l'oreiller: travesseiro\\
        le douche: chuveiro\\
        l'évier: pia\\
        le    robient       : torneira\\
        le    coussin       : almofada\\
        le    toilette      : privada\\
              couloir       : corredor\\
        Frigo, réfrigérateur: geladeira\\
              fenêtre       : janela\\
        l'ascenseur: elevador\\
        sonner   à la porte                : tocar a campanhia\\
        regarder la téve                   : assistir televisão\\
        faire    la vaisselle              : lavar a louça\\
        monter   et descendre les escaliers: subir descer as escadas\\
        aller    aux toilleit              : ir ao banheiro\\
        se       doucher                   : tomar banho\\
        faire    la sieste                 : dormir ou cochilar\\
        s'habiller: se vestir\\
        arroser  les plantes   : molhar as plantas\\
        louer    un appartement: alugar um apartamento\\
        faire    du jardinage  : fazer o jardim\\
        au       dernier étage : último andar\\
                 toit          : telhado\\
                 lavevaisselle : lavalouça\\
                 cuisière      : fogão\\
        produits dientretien   : produtos de limpeza\\
                 lit           : cama\\
                 bureau        : escritório: escrivaninha\\
                 meubles       : móveis\\
                 gazinière     : coifa\\
                 rideaux       : cortinhas\\
        un       évier         : torneira\\
        un       placard       : armário embutido\\
        une      commode       : comoda\\
        une      étagère       : estante\\
        table    de nuit       : criado mudo\\
                 ampoule       : lâmpada\\
        store    vénitien      : persiana\\
        lit      simple        : cama de solteiro\\
        lit      double        : cama de casal\\
        panneau  solaire       : painel solar\\
        porte    savon         : saboneteira\\
                 tableau       : quadro\\
                 matelas       : colchão\\
        le       fauteuil      : poltrona\\
        un       tiroir        : gaveta\\
        salon: living room\\
        pièce: room\\
        
        
    \subsection{Corpo Humano}
        Deus Yeux  : Olhos, Oeil / Olho\\
             Bouche: Boca\\
        Les  Bras  : Braços\\
             Coer  : Coração\\
             tête  : cabeça\\
\end{document}