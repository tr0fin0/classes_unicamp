\documentclass{article}

\usepackage[a4paper, hmargin={20mm, 20mm}, vmargin={25mm, 30mm}]{geometry}
\usepackage[utf8]{inputenc}
\usepackage[english, main=portuguese]{babel}

\usepackage[hidelinks]{hyperref}
\usepackage{bookmark}
\usepackage{cancel}
\usepackage{comment}

\usepackage{array}
\usepackage{indentfirst}
\usepackage{multicol}
\setlength{\multicolsep}{2pt}% 50% of original values
\usepackage{subfiles}

\usepackage{titlesec}

\usepackage{amsmath}
\usepackage{amssymb}
\usepackage{systeme}
\usepackage{float}
\usepackage{enumitem}
\usepackage[thinc]{esdiff} %parcial derivatives
\restylefloat{table}

\usepackage{graphicx}
\usepackage{subcaption}
\graphicspath{ {./images/} }

% Pacote para a definição de novas cores
\usepackage{xcolor}
% Definindo novas cores
\definecolor{darkgreen}{rgb}{0.0, 0.42, 0.24}
\definecolor{darkpurple}{rgb}{0.74, 0.2, 0.64}
\definecolor{darkblue}{rgb}{0.0, 0.28, 0.67}

% Configurando espaços entre paragrafos
%\setlength{\parskip}{0.5em}


% Configurando pacote de circuitos
\usepackage{circuitikz}

%Configurando pacote de Gráficos
\usepackage{tikz}

% Configurando layout para mostrar códigos
\usepackage{listings}

%\titleformat{<command>}[<shape>]{<format>}{<label>}{<sep>}{<before-code>}[<after-code>]
\titleformat
{\section} %comand
[block]  %shape
{\normalfont\LARGE} %format
{\thesection. } %label
{0mm} %sep
{} %before-code
[{\titlerule[0.1mm]}] %after-code

\titlespacing*{\section}{0mm}{0mm}{15mm}

\titleformat
{\subsection} %comand
[block]  %shape
{\normalfont\Large} %format
{\thesubsection. } %label
{0mm} %sep
{} %before-code
[] %after-code

\titlespacing*{\subsection}{0mm}{5mm}{2.5mm}


\begin{document}
    \begin{titlepage}
        \begin{center}
            \rule{450pt}{0.5pt}\\[4mm]
            {\Huge EE532 - Eletrônica Aplicada}\\
            \rule{450pt}{0.5pt}\\[2mm]
            {\Large Resumo de Fórmulas}\\[200mm]
            \today\\
            \rule{250pt}{0.5pt}\\
            {\large Guilherme Nunes Trofino}\\
            {\large 217276}\\
        \end{center}
    \end{titlepage}
\newpage

    \section{test}
    \subsection{Amplificador Filtro Passa Banda Ativo}
        \paragraph{Definição}Amplificador que filtrará o sinal a partir da frequência definida pela combinação de $R_{1}$ e $C_{1}$ na entrada suportanto no máximo a frequência definida pela combinação de $R_{2}$ e $C_{2}$ na saída, sendo obtido pela seguinte configuração:
            \begin{figure}[H]
                \centering
                \begin{circuitikz}[american]
                    \ctikzset{
                        component text=left, 
                        diodes/scale=0.5, 
                        capacitors/scale=0.75, 
                        resistors/width=0.25, 
                        resistors/zigs=1
                    }
                    \draw
                    (0,0) node[op amp, noinv input up] (opamp) {}
    
                    (opamp.+)   to[short, -*] ++(-1.5,0) coordinate (leftR)
                                to[R, l=${R_{1}}$] ++(0,-2)
                                node[tlground] {}

                    (leftR) to[C, l_=${C_{1}}$, -o] ++(-2,0)
                            node[left] {$v_{\text{in}}$}

                    (opamp.out) to[R, l=$R_{3}$, *-] ++(0,-1.5) coordinate (mid)
                                to[R, l=$R_{4}$] ++(0,-1.5)
                                node[tlground] {}
        
                    (opamp.-) --++(0,-1) to[short, -*] (mid)

                    (opamp.out) to[R, l=${R_{2}}$] ++(2, 0) coordinate (out)
                                to[C, l=${C_{2}}$] ++(0,-2)
                                node[tlground] {}

                    (out)   to[short, *-o] ++(0.5,0)
                            node[right] {$v_{\text{out}}$};
                \end{circuitikz}
                \caption{Amplificador Filtro Passa Banda Ativo}
            \end{figure}\noindent
        Note que nesta configuração as seguintes equações serão aplicadas:
            \begin{equation}
                \boxed{
                    f_{\text{Baixa}} = \frac{1}{2\pi R_{1}C_{1}}
                }
                \qquad
                \boxed{
                    f_{\text{Alta}} = \frac{1}{2\pi R_{2}C_{2}}
                }
                \qquad
                \boxed{
                    A = 1 + \frac{R_{3}}{R_{4}}
                }
            \end{equation}
\end{document}