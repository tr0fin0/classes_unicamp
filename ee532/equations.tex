\documentclass{article}

\usepackage[a4paper, hmargin={20mm, 20mm}, vmargin={25mm, 30mm}]{geometry}
\usepackage[utf8]{inputenc}
\usepackage[english, main=portuguese]{babel}

\usepackage[hidelinks]{hyperref}
\usepackage{bookmark}
\usepackage{cancel}
\usepackage{comment}

\usepackage{array}
\usepackage{indentfirst}
\usepackage{multicol}
\setlength{\multicolsep}{2pt}% 50% of original values
\usepackage{subfiles}

\usepackage{titlesec}

\usepackage{amsmath}
\usepackage{amssymb}
\usepackage{systeme}
\usepackage{float}
\usepackage{enumitem}
\usepackage[thinc]{esdiff} %parcial derivatives
\restylefloat{table}

\usepackage{graphicx}
\usepackage{subcaption}
\graphicspath{ {./images/} }

% Pacote para a definição de novas cores
\usepackage{xcolor}
% Definindo novas cores
\definecolor{darkgreen}{rgb}{0.0, 0.42, 0.24}
\definecolor{darkpurple}{rgb}{0.74, 0.2, 0.64}
\definecolor{darkblue}{rgb}{0.0, 0.28, 0.67}

% Configurando espaços entre paragrafos
%\setlength{\parskip}{0.5em}


% Configurando pacote de circuitos
\usepackage{circuitikz}

%Configurando pacote de Gráficos
\usepackage{tikz}

% Configurando layout para mostrar códigos
\usepackage{listings}

%\titleformat{<command>}[<shape>]{<format>}{<label>}{<sep>}{<before-code>}[<after-code>]
\titleformat
{\section} %comand
[block]  %shape
{\normalfont\LARGE} %format
{\thesection. } %label
{0mm} %sep
{} %before-code
[{\titlerule[0.1mm]}] %after-code

\titlespacing*{\section}{0mm}{0mm}{15mm}

\titleformat
{\subsection} %comand
[block]  %shape
{\normalfont\Large} %format
{\thesubsection. } %label
{0mm} %sep
{} %before-code
[] %after-code

\titlespacing*{\subsection}{0mm}{5mm}{2.5mm}


\begin{document}
    \begin{titlepage}
        \begin{center}
            \rule{450pt}{0.5pt}\\[4mm]
            {\Huge EE532 - Eletrônica Aplicada}\\
            \rule{450pt}{0.5pt}\\[2mm]
            {\Large Resumo de Fórmulas}\\[200mm]
            \today\\
            \rule{250pt}{0.5pt}\\
            {\large Guilherme Nunes Trofino}\\
            {\large 217276}\\
        \end{center}
    \end{titlepage}
\newpage

    \section{test}
    \begin{figure}[H]
        \centering
        \begin{circuitikz}[american]
            \ctikzset{
                component text=center, 
                diodes/scale=0.5, 
                capacitors/scale=0.75, 
                resistors/width=0.25, 
                resistors/zigs=1
            }
            \draw
            (0,0)   node[op amp, noinv input up] (U1) {${U_{1}}$}
            (0,2.5) node[op amp, noinv input up] (U2) {${U_{2}}$}
            (0,5)   node[op amp, noinv input up] (U3) {${U_{3}}$}

            (U3.-)  to[short] ++(-0.5,0) coordinate (U3m)
            (U2.-)  to[short] ++(-0.5,0) coordinate (U2m)
            (U1.-)  to[short] ++(-0.5,0)
                    to[short, -*]  (U2m)
                    to[short, -*]  (U3m)
                    to[short] ++(0,3)
                    node[vcc]{${V_{\text{in}}}$}

            (U3.+)  to[short] ++(-1.5,0) coordinate (U3p)
            (U2.+)  to[short] ++(-1.5,0) coordinate (U2p)
            (U1.+)  to[short] ++(-1.5,0) coordinate (U1p)
                    to[R, l_=${R}$] ++(0,-2.5)
                    node[tlground]{}

            (U2p)   to[R, l_=${R}$, *-*] (U1p)
            (U3p)   to[R, l_=${R}$, *-*] (U2p)
            (U3p)   to[R, l =${R}$] ++(0,2)
                    node[vcc]{${V_{\text{CC}}}$}

            (U1p)   node[left]{1.25 V}
            (U2p)   node[left]{2.5 V}
            (U3p)   node[left]{3.75 V}

            (U1.out)node[above]{$A_{1}$}
            (U2.out)node[above]{$A_{2}$}
            (U3.out)node[above]{$A_{3}$}
            
            (U3.out)to[short] ++(0.5,0) coordinate (U3o)
            (U2.out)to[short] ++(0.5,0) coordinate (U2o)
            (U1.out)to[short] ++(0.5,0) coordinate (U1o)
            
            ;
        \end{circuitikz}
    \end{figure}\noindent
\end{document}