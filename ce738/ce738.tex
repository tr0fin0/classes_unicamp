\documentclass{article}
\usepackage{tpack}

\title{CE738 - Economia para Engenharia}
\author{Guilherme Nunes Trofino}
\authorRA{217276}
\project{Resumo Teórico}

\begin{document}
    \maketitle
\newpage

    \tableofcontents
\newpage

    \section{Introdução}
        \paragraph{Apresentação}Neste documento será descrito as informações necessárias para compreensão e solução de exercícios relacionados a disciplina \thetitle . Note que este documento são notas realizadas por \theauthor , em \today.

        \subsection{Sistemas Econômicos}
            \paragraph{Definição}Forma política, social e econômica de organização de uma determinada sociedade que envolverá os seguintes elementos básicos
                \begin{enumerate}[noitemsep, rightmargin = \leftmargin]
                    \item \textbf{Agentes Econômicos:} Responsáveis pela movimentação do mercado através de compra e venda de produtos e serviços;

                    \item \textbf{Instituições:} Responsáveis por oferecer produtos e serviços ou regulando o funcionamento do mercado;

                    \item \textbf{Recursos Produtivos:} Produtos ou serviços disponíveis ao consumo ou a utilização;
                \end{enumerate}
            Normalmente classificados nas seguintes categorias:
                \begin{multicols}{2}
                    \begin{enumerate}[rightmargin = \leftmargin]
                        \item \textbf{Capitalista:} Economia de mercado;
                            \begin{enumerate}[noitemsep]
                                \item \texttt{Propriedade:} Privada dos meios de produção;
                                \item \texttt{Organização:} Regida pelas forças do mercado;
                            \end{enumerate}

                        \columnbreak

                        \item \textbf{Socialista:} Economia centralizada;
                            \begin{enumerate}[noitemsep]
                                \item \texttt{Propriedade:} Coletiva dos meios de produção;
                                \item \texttt{Organização:} Regida por um órgão central;
                            \end{enumerate}
                    \end{enumerate}
                \end{multicols}
            Cada produto ou serviço disponível no mercado terá seu valor determinando por uma série de fatores entre os quais os enunciados abaixo:
                \begin{enumerate}
                    \item \textbf{Raridade:}
                        \begin{enumerate}[noitemsep]
                            \item \texttt{Livres:};
                            \item \texttt{Econômicos:};
                        \end{enumerate}

                    \item \textbf{Natureza:}
                        \begin{enumerate}[noitemsep]
                            \item \texttt{Materiais:};
                            \item \texttt{Imateriais:};
                        \end{enumerate}

                    \item \textbf{Destino:}
                        \begin{enumerate}[noitemsep]
                            \item \texttt{Consumo:};
                            \item \texttt{Capital:};
                            \item \texttt{Intermediários:} Consumidos ao longo da cadeia produtiva;
                        \end{enumerate}

                    \item \textbf{Origem:}
                        \begin{enumerate}[noitemsep]
                            \item \texttt{Privado:};
                            \item \texttt{Público:} Gerenciado por instituições públicas;
                        \end{enumerate}
                \end{enumerate}
\end{document}